\documentclass[handout, aspectratio=169]{beamer}
\mode<presentation>{}
\usepackage[utf8]{inputenc}
\newcommand{\fl}[1]{\left\lfloor #1 \right\rfloor}


\title{MA 105 : Calculus\\ D1 - T5, Tutorial 11}  % change
\author{Aryaman Maithani}
\date[6-11-2019]{6th November, 2019}               % change
\institute[IITB]{IIT Bombay}
\usetheme{Warsaw}
\usecolortheme{beetle}
\newtheorem{defn}{Definition}
\begin{document}
\begin{frame}
	\titlepage
\end{frame}
\begin{frame}{Sheet 12} 
	(3) Compute the surface area of that portion of the sphere $x^{2}+y^{2}+z^{2}=a^{2}$ which lies within the cylinder $x^{2}+y^{2}=a y,$ where $a > 0.$\
	\uncover<2->{There are two pieces of the surface - one below and one above the $xy$-plane, both having the same surface area. }\uncover<3->{Let $S$ be the upper piece. Then one has }
	\uncover<4->{	\[\operatorname{Area}(S) = \iint_T\sqrt{1 + z_x^2 + z_y^2}d(x, y),\]}
	\uncover<5->{where $T$ is the disc }
	\uncover<6->{\[(x, y) \in \mathbb{R}^2 : x^2 + \left(y - \frac{a}{2}\right)^2 \le \left(\frac{a}{2}\right)^2,\] }
	\uncover<7->{and $z(x, y) = \sqrt{a^2 - x^2 - y^2}$ for $(x, y) \in T.$ }
\end{frame}
\begin{frame}{Sheet 12} 
	Now, we calculate $z_x$ and $z_y.$\\
	\uncover<2->{\[z_x = -\frac{x}{z} \text{ and } z_y = -\frac{y}{z}.\] }
	\uncover<3->{Thus, we get the area integral as }
	\uncover<4->{\[\operatorname{Area}(S)=\iint_{T} \frac{a d x d y}{z}=\iint_{T} \frac{a d x d y}{\sqrt{a^{2}-x^{2}-y^{2}}}\] }
	\uncover<5->{Now, $T$ is described in polar coordinates by }
	\uncover<6->{\[x=r \cos \theta, y=r \sin \theta ; 0 \leq \theta \leq \pi, 0 \leq r \leq a \sin \theta.\] }
\end{frame}
\begin{frame}{Sheet 12} 
	Therefore,
	\begin{align*} 
		\uncover<2->{\operatorname{Area}(S)&=\int_{0}^{\pi}\left(\int_{0}^{a \sin \theta} \frac{ar}{\sqrt{a^{2}-r^{2}}}dr\right) d \theta }\\~\\
		\uncover<3->{&=\left.a \int_{0}^{\pi}[-\sqrt{a^{2}-r^{2}}]\right|_{0} ^{a \sin \theta} d \theta }\\~\\
		\uncover<4->{&=a \int_{0}^{\pi}(-a|\cos \theta|+a) d \theta=(\pi-2) a^{2}. }
	\end{align*}
	\uncover<5->{Thus, the required area is $2(\pi-2) a^{2}.$ }
\end{frame}
\begin{frame}{Sheet 12} 
	(6) We shall consider the cylinder (and thus, the sphere) to have radius $1.$ Moreover, we shall choose our axes such that the center of the sphere is the origin and the axis of the cylinder is the $z-$axis.\\
	\uncover<2->{Let us find the area of the sphere between the planes $z = 0$ and $z = h$ for some $h \in (0, 1].$ }\uncover<3->{Using this, we can find the area between two planes according to whether or not they are on the same side of $z = 0$ or not. }\\
	\uncover<4->{The surface of interest is parameterised as: }\\
	\uncover<5->{\[\Phi(\varphi, \theta):=(\sin \varphi \cos \theta, \sin \varphi \sin \theta, \cos \varphi) \text{ where } -\pi \le \theta \le \pi,\; \alpha \le \varphi \le \pi/2,\] }
	\uncover<6->{where $\alpha$ is the (unique) real number in $[0, \pi/2]$ such that $\cos \alpha = c.$ }
\end{frame}
\begin{frame}{Sheet 12} 
	Now, we have $\Phi_{\varphi} \times \Phi_{\theta}=\left(\sin ^{2} \varphi \cos \theta, \sin ^{2} \varphi \sin \theta,sin \varphi \cos \varphi\right).$\\
	\uncover<2->{This gives us $\|\Phi_{\varphi} \times \Phi_{\theta}\| = \sin \varphi.$ }\\
	\uncover<3->{Thus, the area is given by }
	\uncover<4->{\[\int_{-\pi}^{\pi} \int_{\alpha}^{\pi/2} 1\sin\varphi \text{d}\varphi \text{d}\theta\] }
	\[\uncover<5->{ = 2\pi\cos\alpha }\uncover<6->{=2\pi c. }\]
	\uncover<6->{Moreover, it can be easily seen the surface area of the cylinder between these two planes is also given by $2\pi c.$ }\uncover<7->{Thus, we can now conclude the final result by taking two cases. }
\end{frame}
\begin{frame}{Sheet 12} 
	(7) (i) Let $T$ be the region in the $uv-$plane parameterising the region $S$ as given.\\
	\uncover<2->{Note that $\mathbf{r}_{u} \times \mathbf{r}_{v}=-2(\mathbf{i}+\mathbf{j}+\mathbf{k})$ has negative $z-$component. }\uncover<3->{Thus, we get the following, in differential notation: }
	\uncover<4->{\[\mathbf{\hat{n}}dS = \hat{\boldsymbol{n}}\left\|\mathbf{r}_{u} \times \mathbf{r}_{v}\right\| d(u, v) = 2(\mathbf{i}+\mathbf{j}+\mathbf{k}) d(u, v).\] }
	\uncover<5->{Thus, the integral is simply $\displaystyle\iint_T2dS$}\uncover<6->{ $ = 2\operatorname{Area}(T).$ }\\
	\uncover<7->{Note that $T$ is the triangle in the $uv-$plane with vertices $(0, 0)$ and $\left(\frac{1}{2}, \pm \frac{1}{2}\right).$ }\\
	\uncover<8->{Thus, the answer is simply $\frac{1}{2}.$ }

\end{frame}
\begin{frame}{Sheet 12} 
	(ii) The surface satisfies $z = 1 - x - y \ge 0,\;x \ge 0,\;y \ge 0.$\\~\\
	\uncover<2->{Define $T:= \{(x, y) \in \mathbb{R}^2 : x+y \leq 1, x \geq 0, y \geq 0\}.$ }\uncover<3->{Thus, we then have that $S$ is given by $z = f(x, y) := 1 - x -  y$ for $(x, y) \in T.$ }\\~\\
	\uncover<4->{Thus, $\mathbf{\hat{n}}dS = (-z_x, -z_y, 1)d(x, y),$ in differential notation. }\\
	\uncover<5->{Moreover, $\mathbf{F} \cdot \mathbf{n} d S=(x, y, z) \cdot\left(-z_{x},-z_{y}, 1\right) d(x, y)=(x+y+z)d(x, y)=1d(x, y).$ }\\
	\uncover<6->{Now, one has $\displaystyle\iint_{S} \mathbf{F} \cdot \mathbf{n} d S$ } \uncover<7->{$=\displaystyle\iint_{T} 1d(x, y)$ }\uncover<8->{$ = \operatorname{Area}(T) = \frac{1}{2}.$ }
\end{frame}
\begin{frame}{Sheet 12} 
	(8) Routine calculation is to be done.\\
	\uncover<2->{Parameterise the sphere as $\Phi(\varphi, \theta):=(a \sin \varphi \cos \theta, a \sin \varphi \sin \theta, a \cos \varphi)$ for $(\varphi, \theta) \in[0, \pi] \times[-\pi, \pi].$ }\\~\\
	\uncover<3->{Then, $\hat{\boldsymbol{n}} d S = $ }\uncover<4->{$ = \left(\Phi_{\varphi} \times \Phi_{\theta}\right) d(\varphi, \theta)$ }\uncover<5->{$ = (a \sin \varphi) d(\varphi, \theta).$ }\\~\\
	\uncover<4->{(Note that this is indeed the outwards normal.) }\\
	\uncover<5->{The integrand is now $\mathbf{F} \cdot\left(\mathbf{r}_{\theta} \times \mathbf{r}_{\varphi}\right)=a^{4} \sin ^{3} \varphi \cos \varphi\left(1+\cos ^{2} \theta\right).$ }\\
\end{frame}
\begin{frame}{Sheet 12} 
	\uncover<1->{(Check! I might have made a sign mistake.) }\\
	\uncover<2->{Thus, the required integral is }
	\uncover<3->{\[\int_{0}^{2 \pi} \left(\int_{0}^{\pi} a^{4} \sin ^{3} \varphi \cos \varphi\left(1+\cos ^{2} \theta\right) d \theta\right) d \varphi\] }
	\[\uncover<4->{=a^{4}\left(\int_{0}^{\pi} \sin ^{3} \varphi \cos \varphi d \varphi\right)\left(\int_{0}^{2 \pi}\left(1+\cos ^{2} \theta\right) d \theta\right) }\uncover<5->{ = 0. }\]
	
\end{frame}
\begin{frame}{Sheet 13} 
	(2) For $\mathbf{F}=y z \mathbf{i}+x z \mathbf{j}+x y \mathbf{k},$
	\uncover<2->{\[\operatorname{curl}(\mathbf{F})=\left|\begin{array}{ccc}{i} & {j} & {k} \\ {\frac{\partial}{\partial x}} & {\frac{\partial}{\partial y}} & {\frac{\partial}{\partial z}} \\ {y z} & {z x} & {x y}\end{array}\right|=(x-x) \mathbf{i}+(y-y) \mathbf{j}+(z-z) \mathbf{k}=\mathbf{0}.\] }
	\uncover<3->{Let $S$ be any \emph{good enough} (geometric) surface in $\mathbb{R}^3$ such that $\eth S = C.$ }\uncover<4->{Moreover, assume that $S$ is oriented such that orientation it induces on $C$ is the desired orientation. Then, we have that the line integral is given by }
	\[\uncover<4->{\iint_{S} \operatorname{curl}(\mathbf{F}) \cdot \mathbf{n} d S }\uncover<5->{ = 0. }\]
\end{frame}
\begin{frame}{Sheet 13} 
	(3)  By Stokes’ theorem, we have
	\uncover<2->{\[\iint_{S} \operatorname{curl}(\mathbf{v}) \cdot \mathbf{n} d S=\oint_{C_{1}} \mathbf{v} \cdot d \mathbf{s}+\oint_{C_{2}} \mathbf{v} \cdot d \mathbf{s}.\] }
	\uncover<3->{where $C_1$ is the circle $x^{2}+y^{2}=4, z=-3$  with the counterclockwise orientation when viewed from ``high above,'' and $C_2$ is the circle $x^{2}+y^{2}=4, z=0,$ with the opposite orientation. }\\
	\uncover<4->{(How did we decide the orientation?) }\\
	\uncover<5->{Now, we can write $\displaystyle\oint_{C_i}\mathbf{v}\cdot d\mathbf{s}$ as $\displaystyle\oint _{C_i}ydx + xz^3dy - zy^3dz.$ }
\end{frame}
\begin{frame}{Sheet 13} 
	For $i=1,$ we have:\\
	\uncover<2->{$\displaystyle\oint _{C_1}ydx + xz^3dy - zy^3dz$ }\uncover<3->{$ = \displaystyle\oint _{C_1}ydx - 27xdy$ }\uncover<4->{$ =\displaystyle\oint_{C_1}\nabla(xy)\cdot d\mathbf{s} - \oint_{C_1}28y^2dy.$ }\\~\\
	\uncover<3->{The latter integral can be easily evaluated by a suitable parameterisation of $C_1$ to give us $-28 \displaystyle\int_{-\pi}^{\pi} 4 \cos ^{2} \theta d \theta=-112 \pi.$ }\\~\\
	\uncover<4->{Similarly, for $i = 2,$ we have }\\
	\uncover<5->{$\displaystyle\oint_{C_{2}} y d x=-\int_{\pi}^{\pi}(-4 \sin ^{2} \theta ) d \theta=4 \pi.$ }\\~\\
	\uncover<6->{Hence, the required integral is $-108\pi.$ }
\end{frame}
\begin{frame}{Sheet 13} 
	(5) Note the following:
	\uncover<2->{\[\mathbf{F}=\left(y^{2}-z^{2}\right) \mathbf{i}+\left(z^{2}-x^{2}\right) \mathbf{j}+\left(x^{2}-y^{2}\right) \mathbf{k},\] }
	\uncover<3->{\[\nabla\times \mathbf{F}=(-2 y-2 z) \mathbf{i}+(-2 z-2 x) \mathbf{j}+(-2 x-2 y) \mathbf{k},\] }
	\uncover<4->{\[\text{and }\mathbf{\hat{n}}=\frac{\mathbf{i}+\mathbf{j}+\mathbf{k}}{\sqrt{3}}.\] }
	\uncover<5->{Now, along the surface $S$ which is a part of the plane $x + y + z = \frac{3a}{2}$ and which is bounded by $C,$ we have }\\
	\uncover<6->{\[\nabla\times\mathbf{F}\cdot\mathbf{\hat{n}} = -\frac{4}{\sqrt{3}}\frac{3a}{2}.\] }
\end{frame}
\begin{frame}{Sheet 13} 
	Hence, 
	\[\uncover<2->{\iint_{S} \operatorname{curl} \mathbf{F} \cdot \mathbf{n} d S }\uncover<3->{ = -2 \sqrt{3} a \iint_{S} d S }\uncover<4->{ = (-2 \sqrt{3} a)(\text{Area of } S). }\]
	\uncover<3->{The surface S is a regular hexagon with vertices $(a / 2,0, a),(a, 0, a / 2),(a, a / 2,0),(a / 2, a, 0), (0, a, a / 2),(0, a / 2, a).$ }\\
	\uncover<4->{Hence, its area is $\frac{3\sqrt{3}}{4}a^2.$ }\\
	\uncover<5->{Using Stokes' theorem, we get that the above integral is equal to our desired integral which comes out to be $-\frac{9 a^{3}}{2}.$ }
\end{frame}
\begin{frame}{Sheet 13} 
	(6) Consider the following:
	\begin{align*} 
		\uncover<2->{\mathbf{F} &= (y, z, x), }\\
		\uncover<3->{\nabla\times\mathbf{F}&=-(1, 1, 1). }
	\end{align*}
	\uncover<4->{We are to compute $\displaystyle\oint_C\mathbf{F}\cdot d\mathbf{s}.$ }\\
	\uncover<5->{Let $S$ be the surface lying on the hyperboloid bounded by $C.$ We shall now describe $S.$ }\\
	\uncover<6->{Let $D := \{(x, y) \in \mathbb{R}^2 : x^2 + y^2 \le a^2\}.$ }\\~\\
	\uncover<7->{Let $f(x, y) := \dfrac{xy}{b}$ for $(x, y) \in D.$ }\\
\end{frame}
	
\begin{frame}{Sheet 13} 
	\uncover<1->{Then, the surface $S$ is given by $z = f(x, y).$ }\uncover<2->{Following differential notation, we get $\mathbf{\hat{n}}dS = (-z_x, -z_y, 1)d(x, y) = (-\frac{y}{b}, -\frac{x}{b}, 1)d(x, y).$}\\
	\uncover<3->{By Stokes' Theorem, the required integral is simply equal to $\displaystyle\dfrac{1}{b}\iint_D(y + x - b)d(x, y).$ }\\
	\uncover<4->{This can be easily solved using polar coordinates. }
\end{frame}
\begin{frame}{Farewell}                            % change
	
\end{frame}
\end{document}	