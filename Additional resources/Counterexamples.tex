\documentclass{article}
\usepackage{amsmath, amssymb, amsfonts, amsthm, mathtools}
\usepackage[utf8]{inputenc}
\usepackage[inline]{enumitem}
\usepackage{cancel}
\usepackage{soul}
\usepackage{hyperref}
\usepackage{centernot}

\setlength\parindent{0pt}
\let\emptyset\varnothing

\usepackage{xcolor}
\definecolor{mybgcolor}{RGB}{50, 50, 50} %46, 51, 63

\usepackage{pagecolor}
\pagecolor{mybgcolor}
\color{white}

\usepackage{titlesec}
\titleformat{\section}[block]
  {\normalfont\scshape}{}{0.25cm}{\large}

\usepackage{geometry}
\geometry{
	a4paper,
	total={170mm,257mm},
	left=20mm,
	top=20mm,
}

\title{Counterexamples in Calculus}
\author{Aryaman Maithani\\\small TA of D1-T5}
\date{Semester: Autumn 2019\\ Latest update: \today}

\begin{document}
\maketitle
\hrulefill
\begin{enumerate} 
	\item A bounded sequence need not be convergent.\\
	\textbf{Example:} $a_n := (-1)^n.$
	%
	\item A continuous function need not have the intermediate value property.\\
	\textbf{Example:} $f:(0, 1) \cup (2, 3) \to \mathbb{R}$ given by $f(x) := x.$
	\item The inverse of a differentiable function need not be continuous.\\
	\textbf{Example:} $f:[0, 1] \cup (2, 3] \to [0, 2]$ given by
	\[f(x) := \left\{
	\begin{array}{l c}
		x & x \in [0, 1]\\
		x-1 & x \in (2, 3]		
	\end{array}
	\right.\]
	Corollaries: The inverse of a continuous function need not be continuous. The inverse of a differentiable function need not be differentiable.
	\item A function defined on an interval with the intermediate value property need not be continuous \emph{anywhere.}\\
	\textbf{Example:} Conway Base 13 function.
	\item A Riemann integrable function may have infinitely many discontinuities.\\
	\textbf{Example:} Thomae's functions.
	\item A differentiable function with derivative zero everywhere need not be constant.\\
	\textbf{Example:} $f:(0, 1) \cup (2, 3) \to \mathbb{R}$ defined as
	\[f(x) := \left\{
	\begin{array}{l c}
		1 & x \in (0, 1)\\
		2 & x \in (2, 3)		
	\end{array}
	\right.\]
	\item A differentiable function with strictly negative derivative everywhere need not be monotonically decreasing.\\
	\textbf{Example:} $f:\mathbb{R}\setminus\{0\} \to \mathbb{R}$ defined as $f(x) := x^{-1}.$
	%
	\item \label{diric} Let $f:[0, 1]\to\mathbb{R}$ be a bounded function. $f$ need not be integrable on $[0, 1].$\\
	\textbf{Example:} Take $f$ to be the Dirichlet function defined as
	\[f(x) := \left\{
	\begin{array}{r c}
		1 & x \in \mathbb{Q}\\
		0 & x \not\in \mathbb{Q}		
	\end{array}
	\right.\]
	%
	\item Integrability of $|f|$ does not imply integrability of $f.$\\
	\textbf{Example:} $f:[-1, 1] \to \mathbb{R}$ defined as
	\[f(x) := \left\{
	\begin{array}{r c}
		1 & x \in \mathbb{Q}\\
		-1 & x \not\in \mathbb{Q}		
	\end{array}
	\right.\]
	%
	\item Let $f:\mathbb{R}^2\to\mathbb{R}$ be a differentiable function such that $f_{xy}(0, 0)$ and $f_{yx}(0, 0)$ exist. It is not necessary that they are equal.\\
	\textbf{Example:} Let $f:\mathbb{R}^2 \to \mathbb{R}$ be defined as
	\[f(x, y) := \left\{
	\begin{array}{c c}
		\dfrac{xy(x^2 - y^2)}{x^2 + y^2} & (x, y) \neq (0, 0)\\
		0 & (x, y) = (0, 0)	
	\end{array}
	\right.\]
	%
	\item Suppose $f:\mathbb{R}^2\to\mathbb{R}$ is a function such that all directional derivatives of $f$ exist at $(0, 0).$ It is not necessary that the function is continuous at $(0, 0).$\\
	\textbf{Example:} Let $f:\mathbb{R}^2 \to \mathbb{R}$ be defined as
	\[f(x, y) := \left\{
	\begin{array}{c c}
		\dfrac{x^3y}{x^6 + y^2} & (x, y) \neq (0, 0)\\
		0 & (x, y) = (0, 0)	
	\end{array}
	\right.\]
	The above function also satisfies $(\mathbf{D_u}f)(x_0, y_0) = (\nabla f)(x_0, y_0)\cdot\mathbf{u}$ for every unit vector $\mathbf{u} \in \mathbb{R}^2.$
	%
	\item Let $(x_0, y_0)$ be an interior point of a subset $D$ of $\mathbb{R}^2,$ and let $f:D\to\mathbb{R}.$ Suppose the following conditions hold:
	\begin{enumerate}[nosep] 
		\item Both partial derivatives $f_x(x_0,y_0)$ and $f_y(x_0, y_0)$ exist.
		\item The directional derivative $(\mathbf{D_u}f)(x_0, y_0)$ exists for every unit vector $\mathbf{u} \in \mathbb{R}^2.$
		\item $(\mathbf{D_u}f)(x_0, y_0) = (\nabla f)(x_0, y_0)\cdot\mathbf{u}$ for every unit vector $\mathbf{u} \in \mathbb{R}^2.$
		\item $f$ is continuous at $(x_0, y_0).$
	\end{enumerate}
	It is not necessary that $f$ is differentiable at $(x_0, y_0).$\\
	\textbf{Example:} Let $f:\mathbb{R}^2 \to \mathbb{R}$ be defined as
	\[f(x, y) := \left\{
	\begin{array}{c c}
		\dfrac{x^3y}{x^4 + y^2} & (x, y) \neq (0, 0)\\
		0 & (x, y) = (0, 0)	
	\end{array}
	\right.\]
	%
	\item Let $D \subset \mathbb{R}^2.$ It is not necessary that $\partial D$ is of content zero or that it is ``one dimensional.''\\
	\textbf{Example:} Let $D = (\mathbb{Q}\cap[0, 1])^2.$ Then, $\partial D = [0, 1]^2.$
	%
	\item Let $R = [a, b] \times[c, d]$ be a closed and bounded rectangle in $\mathbb{R}^2.$ Suppose that $f:R\to\mathbb{R}$ is a function such that the iterated integrals $\displaystyle\int_{a}^{b} \left(\int_{c}^{d} f(x, y) dy\right) dx $ and $\displaystyle\int_{c}^{d} \left(\int_{a}^{b} f(x, y) dx\right) dy $ both exist.\\~\\
	It is not necessary that they are equal.\\
	\textbf{Example:} Let $a = c = - 1$ and $b = d = 1$ and let $f:R \to \mathbb{R}$ be defined as
	\[f(x, y) := \left\{
	\begin{array}{c c}
		\dfrac{2xy(x^2 - y^2)}{(x^2 + y^2)^3} & (x, y) \neq (0, 0)\\
		0 & (x, y) = (0, 0)	
	\end{array}
	\right.\]

	%
	\item Let $R = [a, b] \times[c, d]$ be a closed and bounded rectangle in $\mathbb{R}^2.$ Suppose that $f:R\to\mathbb{R}$ is a function such that the iterated integrals $\displaystyle\int_{a}^{b} \left(\int_{c}^{d} f(x, y) dy\right) dx $ and $\displaystyle\int_{c}^{d} \left(\int_{a}^{b} f(x, y) dx\right) dy $ both exist and are equal.\\~\\
	 It is not necessary that $f$ is Riemann integrable on $R.$\\
	\textbf{Example:} Let $a = c = - 1$ and $b = d = 1$ and let $f:R \to \mathbb{R}$ be defined as
	\[f(x, y) := \left\{
	\begin{array}{c c}
		\dfrac{2xy}{(x^2 + y^2)^2} & (x, y) \neq (0, 0)\\
		0 & (x, y) = (0, 0)	
	\end{array}
	\right.\]
	%
%	\item Let $R = [a, b] \times[c, d]$ be a closed and bounded rectangle in $\mathbb{R}^2.$ Suppose that $f:R\to\mathbb{R}$ is a function such that the iterated integrals $\displaystyle\int_{a}^{b} \left(\int_{c}^{d} f(x, y) dy\right) dx $ and $\displaystyle\int_{c}^{d} \left(\int_{a}^{b} f(x, y) dx\right) dy $ both exist and are equal. \\~\\
%	Moreover, suppose that $f$ is bounded on $R.$\\
%	 It is not necessary that $f$ is Riemann integrable on $R.$\\
%	\textbf{Example:} \\
%	We have a slightly more involved example in this case.\\
%	Let $Q = \mathbb{Q}\cap[0, 1]$ denote the rational numbers in $[0, 1],$ and $I = [0, 1] \times [0, 1]$ denote the unit square in $\mathbb{R}^2.$ For $x \in Q,$ write $x = p/q$ in lowest terms (that is, $0\le p \le q \neq 0$ such that $\gcd(p, q) = 1$), and define the set $S(x) \subset I$ as
%	\[S(x) := \{(m/q, n/q) : n = 0, 1, \ldots, p,\;m = 0, 1, \ldots, p\}.\]
%	We define $S \subset I$ by
%	\[S = \bigcup_{x \in Q}S(x),\]
%	and $f:I \to \mathbb{R}$ by
%	\[f(x, y) := \left\{
%	\begin{array}{c c}
%		0 & (x, y) \in S\\
%		1 & (x, y) \notin S	
%	\end{array}
%	\right.\]
%	In this case, both the iterated integrals
%	\[\int_{0}^{1}\left(\int_{0}^{1} f(x, y) d y\right) d x,\; \int_{0}^{1}\left(\int_{0}^{1} f(x, y) d x\right) d y\]
%	exist and are equal to $1,$ but $f$ is not Riemann integrable on $I.$
	%
	\item \label{iterinte} Let $R = [a, b] \times[c, d]$ be a closed and bounded rectangle in $\mathbb{R}^2.$ Suppose that $f:R\to\mathbb{R}$ is a function such that $f$ is Riemann integrable on $R.$ It is not necessary that both the iterated integrals exist.\\
	\textbf{Example:} Let $a = c = 0$ and $b = d = 1.$ Define $f:R\to\mathbb{R}$ as
	\[f(x, y) := \left\{
	\begin{array}{c l}
		n^{-1} & \text{if }(x, y) \in \mathbb{Q}^2 \text{ and } x = m/n \text{ such that } (m, n) \text{ in simplest form} \\
		0 & (x, y) \notin \mathbb{Q}^2	
	\end{array}
	\right.\]
	(Note that $0/1$ is the simplest form for $0.$)\\~\\
	Then, $f$ is integrable on $R.$ (May be tough to show but it's true.)\\
	However, if you fix $x_0 \in \mathbb{Q}\cap[0, 1],$ then $f(x_0, y)$ is not Riemann integrable (as a function of $y$) on $[0, 1].$ It can be seen easily that it behaves like the Dirichlet function mentioned in point \ref{diric}.\\
	Thus, the iterated integral $\displaystyle\int_{0}^{1} \left(\int_{0}^{1} f(x, y) dy \right) dx $ does not exist.\\
	Note the other iterated integral, however, does exist.
	\item Let $R = [a, b] \times[c, d]$ be a closed and bounded rectangle in $\mathbb{R}^2.$ Suppose $f:R\to\mathbb{R}$ is a function such that one of the iterated integrals exists. It is not necessary that the other does too.\\
	\textbf{Example:} Example \ref{iterinte}.
\end{enumerate}
\end{document}