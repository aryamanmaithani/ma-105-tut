\documentclass[handout, aspectratio=169]{beamer}
\mode<presentation>{}
\usepackage[utf8]{inputenc}
\newcommand{\fl}[1]{\left\lfloor #1 \right\rfloor}


\title{MA 105 : Calculus\\ D1 - T5, Tutorial 09}  % change
\author{Aryaman Maithani}
\date[29-09-2019]{29th September, 2019\\ \tiny (Happy birthday to me!)}               % change
\institute[IITB]{IIT Bombay}
\usetheme{Warsaw}
\usecolortheme{beetle}
\newtheorem{defn}{Definition}
\begin{document}
\begin{frame}
	\titlepage
\end{frame}
\begin{frame}{Summary} 
	Sheet 6: Problems 7, 9, 10
	Sheet 7: Problems 1, 2, 3, 4
\end{frame}
\begin{frame}{An example}
	Consider the function $f:\mathbb{R}^2 \to \mathbb{R}$ defined as follows:
	\[f(x,\;y) := \left\{
	\begin{array}{c c}
		\frac{x^2y}{x^4 + y^2} & (x,\; y) \neq (0,\;0)\\
		0 & (x,\;y) = (0,\;0)
	\end{array}
	\right.\]
	Then, verify that
	\[(\mathbf{D_{u}}f)(0,\;0) = \left\{
	\begin{array}{c c}
		u_1^2/u_2 & u_2 \neq 0\\
		0 & u_2 = 0
	\end{array}
	\right.\]
	Thus, observe that \emph{all} directional derivatives of $f$ at $(0,\;0)$ exist even though $f$ is discontinuous at $(0,\;0).$ Also, note that $f$ is not differentiable, that is, its total derivative does not exist. The last note should not be surprising as $f$ is discontinuous at the origin. Also, note that $(\mathbf{D_u}f)(0,\;0) = (\nabla f)(0,\;0)\cdot\mathbf{u}$ is not true in general.\\~\\
	The aim of this example was to drive home the point that even existence of all directional derivatives does not imply differentiability (or even continuity!).
\end{frame}
\begin{frame}{Sheet 6}
	(7) Argue about the continuity of $f$ at $(0,\;0)$ using the fact that $|f(x,\;y)| \le |x^2 + y^2|.$ (Recall Tutorial 2, Question 3. (ii))\\~\\
	It can also be easily verified that $f_x(0,\;0) = f_y(0,\;0) = 0.$ (Write the expression like the previous questions and arrive at the conclusion.)\\~\\
	Now, let us evaluate $f_x(x_0,\;y_0)$ for $(x_0,\;y_0) \neq (0,\;0).$\\
	It can be easily evaluated using product and chain rules to be:
	\[f_x(x_0,\;y_0) = 2x\left(\sin\left(\frac{1}{x^2 + y^2}\right) - \frac{1}{x^2 + y^2}\cos\left(\frac{1}{x^2 + y^2}\right)\right).\]
	The function $\displaystyle2x\sin\left(\frac{1}{x^2 + y^2}\right)$ is bounded in any disc centered at $(0,\;0).$ \hfill (How?)\\
\end{frame}
	
\begin{frame}{Sheet 6}
	However, $\displaystyle\frac{2x}{x^2 + y^2}\cos\left(\frac{1}{x^2 + y^2}\right)$ is not bounded in any such disc. To see this, consider any $r > 0$ and any $M \in \mathbb{R}.$ One can find an $n \in \mathbb{N}$ such that $\frac{1}{\sqrt{n\pi}} < r$ and $\sqrt{n\pi} > M.$ \hfill (How? Archimedean.)\\
	In that case, the point $(x_0,\;y_0) = (1/\sqrt{2n\pi},\;0)$ will lie in the disc centered at $(0,\;0)$ with radius $r$ and $f(x_0,\;y_0) > M.$\\~\\
	As the sum of a bounded function and an unbounded function is unbounded, we have proven the result.
\end{frame}

\begin{frame}{Sheet 6}
	(9) (i) Let $f:\mathbb{R}^2 \to \mathbb{R}$ denote the function given in the question.\\
	For a unit vector $\textbf{u} := (u_1, u_2)$ and $t \neq 0,$
	\[\frac{f\left(0+t u_{1}, 0+t u_{2}\right)-f(0,0)}{t} = u_1u_2(u_1^2 - u_2^2)t.\]
	Hence, $\left(\mathbf{D_u} f\right)(0,0)$ exists and equals $0$ for all $\textbf{u}.$ Thus, all directional derivatives exist.\\~\\
	\emph{If} $f$ is differentiable, then the total derivative \emph{must} be $(f_x(0, 0), f_y(0, 0)) = (0, 0).$ Let us now see whether this does indeed satisfy the condition for being the total derivative. For that, we must check whether
	\[\lim _{(h, k) \rightarrow(0,0)} \frac{f\left(0+h, 0+k\right)-f\left(0, 0\right)-0 h-0 k}{\sqrt{h^{2}+k^{2}}}=0.\]
\end{frame}

\begin{frame}{Sheet 6}
	For $(h,k) \neq (0,0),$ we have it that
	\[\frac{f\left(0+h, 0+k\right)-f\left(0, 0\right)-0 h-0 k}{\sqrt{h^{2}+k^{2}}} = hk\frac{(h^2 - k^2)}{(h^2 + k^2)^{3/2}}.\]
	Also, note that
	\[\left|hk\frac{(h^2 - k^2)}{(h^2 + k^2)^{3/2}}\right| \le \left|h\frac{k}{\sqrt{h^2 + k^2}}\right| \le |h|.\]
	Thus, the required limit indeed does exist and equals $0.$\\
	Hence, $f$ is differentiable at $(0,0)$ with (total) derivative equal to $(0, 0).$
\end{frame}
\begin{frame}{Sheet 6}
	(9) (ii) Let $f:\mathbb{R}^2 \to \mathbb{R}$ denote the function given in the question.\\
	For a unit vector $\textbf{u} := (u_1, u_2)$ and $t \neq 0,$
	\[\frac{f\left(0+t u_{1}, 0+t u_{2}\right)-f(0,0)}{t} = u_1^3.\]
	Hence, $\left(\mathbf{D_u} f\right)(0,0)$ exists and equals $u_1^3$ for all $\textbf{u}.$ Thus, all directional derivatives exist.\\~\\
	\emph{If} $f$ is differentiable, then the total derivative \emph{must} be $(f_x(0, 0), f_y(0, 0)) = (1, 0).$ Let us now see whether this does indeed satisfy the condition for being the total derivative. For that, we must check whether
	\[\lim _{(h, k) \rightarrow(0,0)} \frac{f\left(0+h, 0+k\right)-f\left(0, 0\right)-1 h-0 k}{\sqrt{h^{2}+k^{2}}}=0.\]
\end{frame}
\begin{frame}{Sheet 6}
	For $(h,k) \neq (0,0),$ we have it that
	\[\frac{f\left(0+h, 0+k\right)-f\left(0, 0\right)-0 h-0 k}{\sqrt{h^{2}+k^{2}}} = -\frac{hk^2}{(h^2 + k^2)^{3/2}}.\]
	It can be seen that the limit for the above expression as $(h, k) \to (0, 0)$ does not exist. Indeed, if one approaches $(0, 0)$ along the curve $h = mk,$ the limit along that path turns out to be $-m/(1 + m^2)^{3/2}.$ Thus, taking $m = 1$ and $m = 0$ demonstrates the non-existence of limit.
\end{frame}
\begin{frame}{Sheet 6}
	(9) (iii) Let $f:\mathbb{R}^2 \to \mathbb{R}$ denote the function given in the question.\\
	For a unit vector $\textbf{u} := (u_1, u_2)$ and $t \neq 0,$
	\[\frac{f\left(0+t u_{1}, 0+t u_{2}\right)-f(0,0)}{t} = t\sin\left(\frac{1}{t^2}\right).\]
	Hence, $\left(\mathbf{D_u} f\right)(0,0)$ exists and equals $0$ for all $\textbf{u}.$ \hfill (How?)\\
	Thus, all directional derivatives exist.\\~\\
	\emph{If} $f$ is differentiable, then the total derivative \emph{must} be $(f_x(0, 0), f_y(0, 0)) = (0, 0).$ Let us now see whether this does indeed satisfy the condition for being the total derivative. For that, we must check whether
	\[\lim _{(h, k) \rightarrow(0,0)} \frac{f\left(0+h, 0+k\right)-f\left(0, 0\right)-0 h-0 k}{\sqrt{h^{2}+k^{2}}}=0.\]
\end{frame}
\begin{frame}{Sheet 6}
	For $(h,k) \neq (0,0),$ we have it that
	\[\frac{f\left(0+h, 0+k\right)-f\left(0, 0\right)-0 h-0 k}{\sqrt{h^{2}+k^{2}}} = \sqrt{h^2 + k^2}\sin\left(\frac{1}{h^2 + k^2}\right).\]
	Also, note that
	\[\left|\sqrt{h^2 + k^2}\sin\left(\frac{1}{h^2 + k^2}\right)\right| \le \left|\sqrt{h^2 + k^2}\right|.\]
	Thus, the required limit indeed does exist and equals $0.$\\
	Hence, $f$ is differentiable at $(0,0)$ with (total) derivative equal to $(0, 0).$
\end{frame}

\begin{frame}{Sheet 6}
	(10) The continuity of $f$ at $(0, 0)$ is easy to show using the $\epsilon-\delta$ condition.\\
	Indeed, observe that $|f(x, y) - f(0, 0)| = \left|\sqrt{x^2 + y^2}\right|$ for $y \neq 0$ and $|f(x, y) - f(0, 0)| = 0$ for $y = 0.$\\
	Thus, in general, we have that $|f(x, y) - f(0, 0)| \le \left|\sqrt{x^2 + y^2}\right|.$ \\
	Let $\delta := \epsilon$ and call it a day.\\~\\
	%
	For a unit vector $\textbf{u} := (u_1, u_2)$ and $t \neq 0,$
	\[\frac{f\left(0+t u_{1}, 0+t u_{2}\right)-f(0,0)}{t} = \left\{
		\begin{array}{c c}
			0 & u_2 = 0\\
			\frac{u_2}{|u_2|} & u_2 \neq 0
		\end{array}
	\right.\]
	Hence, $\left(\mathbf{D_u} f\right)(0,0)$ exists for all $\textbf{u}.$ Thus, all directional derivatives exist.\\~\\
\end{frame}
\begin{frame}{Sheet 6}
	\emph{If} $f$ is differentiable, then the total derivative \emph{must} be $(f_x(0, 0), f_y(0, 0)) = (0, 1).$ Let us now see whether this does indeed satisfy the condition for being the total derivative. For that, we must check whether
	\[\lim _{(h, k) \rightarrow(0,0)} \frac{f\left(0+h, 0+k\right)-f\left(0, 0\right)-0 h- 1k}{\sqrt{h^{2}+k^{2}}}=0.\]
	For $(h,k) \neq (0,0),$ we have it that
	\[\frac{f\left(0+h, 0+k\right)-f\left(0, 0\right)-0 h- 1k}{\sqrt{h^{2}+k^{2}}} = \frac{k}{|k|} - \dfrac{k}{\sqrt{h^2 + k^2}}.\]
	It is clear that the limit of the above expression as $(h, k) \to (0, 0)$ does not exist. (Consider the paths $k = mh.$) Hence, $f$ is not differentiable at $(0, 0).$
\end{frame}

\begin{frame}{Sheet 7}
	(1) Note that the partial derivatives of $F$ do exist at $(1, -1, 3).$ Indeed, given any $(x_0, y_0, z_0),$ we have it that,
	\[F_x(x_0, y_0, z_0) = 2x_0 + 2y_0,\]
	\[F_y(x_0, y_0, z_0) = 2x_0 - 2y_0,\]
	\[F_z(x_0, y_0, z_0) = 2z_0.\]
	Thus, $(\nabla F)(1, -1, 3) = (0, 4, 6).$ \\
	Moreover, the direction of the normal at to the surface $F(x, y, z) = c$ at the point $(x_0, y_0, z_0)$ is given by $(\nabla F)(x_0, y_0, z_0).$ \hfill (How?)\\
	Thus, the required normal line is $(1, -1, 3) + t(0, 4, 6)$ as $t$ varies over $\mathbb{R}.$\\~\\
	Also, the corresponding tangent plane is given by
	\[0\cdot(x - 1) + 4(y + 1) + 6(z - 3) = 0.\]
\end{frame}

\begin{frame}{Sheet 7}
	(2) It is not too tough to show that the direction of the normal to a sphere at a point on the sphere is the same as the direction of the vector joining the center to that point.\\
	Indeed, we get that $(\nabla S)(x_0, y_0, z_0) = 2(x_0, y_0, z_0),$ where $S(x, y, z) := x^2 + y^2 + z^2$ for $(x, y, z) \in \mathbb{R}^3.$\\
	Thus, the required $\mathbf{u}$ is $\frac{1}{3}(2, 2, 1).$\\
	Hence,
	\[(\mathbf{D_u}F)(2, 2, 1) = \lim_{t\to 0}\frac{3(2t/3) - 5(2t/3) + 2(t/3)}{t} = -\frac{2}{3}.\]
\end{frame}

\begin{frame}{Sheet 7}
	(3) We shall assume that $z$ is a ``sufficiently smooth'' function of $x$ and $y.$\\
	We are given that $\sin (x+y)+\sin (y+z)=1$ and $\cos (y+z) \neq 0.$\\ 
	Differentiating with respect to $x$ while keeping $y$ constant gives us $\cos (x+y)+\cos (y+z) \frac{\partial z}{\partial x}=0.$ \hfill $(*)$\\~\\
	Similarly, differentiating with respect to $y$ while keeping $x$ constant gives us $\cos (x+y)+\cos (y+z)\left(1+\frac{\partial z}{\partial y}\right)=0.$ \hfill $(**)$\\~\\
	Differentiating $(*)$ with respect to $y$ gives us $-\sin (x+y)-\sin (y+z)\left(1+\frac{\partial z}{\partial y}\right) \frac{\partial z}{\partial x}+\cos (y+z) \frac{\partial^{2} z}{\partial x \partial y}=0.$ \footnote{Note that I have implicitly assumed that $\frac{\partial^2z}{\partial x\partial y} = \frac{\partial^2z}{\partial y\partial x}.$ However, using a different set of calculations, one can arrive at the same answer without assuming this. I encourage you to try that.}
\end{frame}
\begin{frame}{Sheet 7}
	Thus, using $(*)$ and $(**),$ we get
	\begin{align*} \frac{\partial^{2} z}{\partial x \partial y} &=\frac{1}{\cos (y+z)}\left[\sin (x+y)+\sin (y+z) \cdot\left(1+\frac{\partial z}{\partial y}\right) \frac{\partial z}{\partial x}\right] \\~\\ &=\frac{1}{\cos (y+z)}\left[\sin (x+y)+\sin (y+z)\left(-\frac{\cos (x+y)}{\cos (y+z)}\right)\left(-\frac{\cos (x+y)}{\cos (y+z)}\right)\right] \\~\\ &=\frac{\sin (x+y)}{\cos (y+z)}+\tan (y+z) \frac{\cos ^{2}(x+y)}{\cos ^{2}(y+z)} \end{align*}
\end{frame}

\begin{frame}{Sheet 7}
	(4) We have that
	\[f_{x y}(0,0)=\lim _{k \rightarrow 0} \frac{f_{x}(0, k)-f_{x}(0,0)}{k}.\]
	For $k \neq 0,$ we know that
	\[f_{x}(0, k)=\lim _{h \rightarrow 0} \frac{f(h, k)-f(0, k)}{h}=-k.\]
	We also know that 
	\[f_{x}(0,0)=\lim _{h \rightarrow 0} \frac{f(h, 0)-f(0,0)}{h}=0.\]
\end{frame}

\begin{frame}{Sheet 7}
	Thus, we get that
	\[f_{x y}(0,0)=\lim _{k \rightarrow 0} \frac{-k-0}{k}=-1.\]
	By similar calculations, we get that $f_{y x}(0,0)=1.$\\
	Thus, $f_{x y}(0,0) \neq f_{y x}(0,0).$\\~\\
	For $(x,\;y) \neq (0,\;0),$ one can calculate the second derivatives and see that they turn out to be discontinuous at $(0,\;0).$
	\[{f_{x}(x, y)=\frac{x^{4} y+4 x^{2} y^{3}-y^{5}}{\left(x^{2}+y^{2}\right)^{2}}, f_{y}(x, y)=\frac{x^{5}-4 x^{3} y^{2}-x y^{4}}{\left(x^{2}+y^{2}\right)^{2}}} \]
	\[ {f_{xy}(x,y)=\frac{x^{6}+9 x^{4} y^{2}-9 x^{2} y^{4}-y^{6}}{\left(x^{2}+y^{2}\right)^{3}},f_{yx}(x, y)=\frac{x^{6}+9 x^{4} y^{2}-9 x^{2} y^{4}-y^{6}}{\left(x^{2}+y^{2}\right)^{3}}}\]
\end{frame}
\end{document}