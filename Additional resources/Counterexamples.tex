\documentclass{article}
\usepackage{amsmath, amssymb, amsfonts, amsthm, mathtools}
\usepackage[utf8]{inputenc}
\usepackage[inline]{enumitem}
\usepackage{cancel}
\usepackage{soul}
\usepackage{hyperref}
\usepackage{centernot}

\setlength\parindent{0pt}
\let\emptyset\varnothing

\usepackage{xcolor}
\definecolor{mybgcolor}{RGB}{50, 50, 50} %46, 51, 63

\usepackage{pagecolor}
\pagecolor{mybgcolor}
\color{white}

\usepackage{titlesec}
\titleformat{\section}[block]
  {\normalfont\scshape}{}{0.25cm}{\large}

\usepackage{geometry}
\geometry{
	a4paper,
	total={170mm,257mm},
	left=20mm,
	top=20mm,
}

\title{Counterexamples in Calculus}
\author{Aryaman Maithani}
\date{Semester: Autumn 2019\\ Latest update: \today}

\begin{document}
\maketitle
\hrulefill
\begin{enumerate} 
	\item A bounded sequence need not be convergent.\\
	\textbf{Example:} $a_n := (-1)^n.$
	%
	\item A continuous function need not have the intermediate value property.\\
	\textbf{Example:} $f:(0, 1) \cup (2, 3) \to \mathbb{R}$ given by $f(x) := x.$
	\item The inverse of a differentiable function need not be continuous.\\
	\textbf{Example:} $f:[0, 1] \cup (2, 3] \to [0, 2]$ given by
	\[f(x) := \left\{
	\begin{array}{l c}
		x & x \in [0, 1]\\
		x-1 & x \in (2, 3]		
	\end{array}
	\right.\]
	Corollaries: The inverse of a continuous function need not be continuous. The inverse of a differentiable function need not be differentiable.
	\item A function defined on an interval with the intermediate value property need not be continuous \emph{anywhere.}\\
	\textbf{Example:} Conway Base 13 function.
	\item A Riemann integrable function may have infinitely many discontinuities.\\
	\textbf{Example:} Thomae's functions.
	\item A differentiable function with derivative zero everywhere need not be constant.\\
	\textbf{Example:} $f:(0, 1) \cup (2, 3) \to \mathbb{R}$ defined as
	\[f(x) := \left\{
	\begin{array}{l c}
		1 & x \in (0, 1)\\
		2 & x \in (2, 3)		
	\end{array}
	\right.\]
	\item A differentiable function with strictly negative derivative everywhere need not be monotonically decreasing.\\
	\textbf{Example:} $f:\mathbb{R}\setminus\{0\} \to \mathbb{R}$ defined as $f(x) := x^{-1}.$
	\item Integrability of $|f|$ does not imply integrability of $f.$\\
	\textbf{Example:} $f:[-1, 1] \to \mathbb{R}$ defined as
	\[f(x) := \left\{
	\begin{array}{r c}
		1 & x \in \mathbb{Q}\\
		-1 & x \not\in \mathbb{Q}		
	\end{array}
	\right.\]
\end{enumerate}
\end{document}