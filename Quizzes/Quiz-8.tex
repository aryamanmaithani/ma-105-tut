\documentclass{article}
\usepackage{amsmath, amssymb, amsfonts, amsthm, mathtools}
\usepackage[utf8]{inputenc}
\usepackage[inline]{enumitem}
\usepackage{cancel}
\usepackage{soul}
\usepackage{hyperref}
\newtheorem{theorem}{Theorem}
\newtheorem{lem}{Lemma}
\newtheorem{defn}{Definition}

\setlength\parindent{0pt}

\usepackage{xcolor}
\definecolor{mybgcolor}{RGB}{46, 51, 63} %46, 51, 63

\usepackage{pagecolor}
\pagecolor{mybgcolor}
\color{white}

\usepackage{geometry}
\geometry{
	a4paper,
	total={170mm,257mm},
	left=20mm,
	top=20mm,
}

\title{Short Quiz 8: Solution}      % here
\author{Aryaman Maithani}
\date{29th September, 2019 \\ \tiny (Happy birthday to me!)}  		 % here

\begin{document}
\maketitle

\hrulefill

\textbf{Question.} Consider the function $f:\mathbb{R}^2 \to \mathbb{R}$ defined as follows: $f(x, y) = (x^4 + y^3)/(x^4 + y^2)$ if $(x, y) \neq (0, 0)$ and $f(0, 0) = 0.$ \\
State whether the following statements are true or false. Justify your answer.
\begin{enumerate}[label = (\alph*)] 
	\item The function $f$ is continuous at $(0,0).$
	\item Both the partial derivatives of $f$ exist.
\end{enumerate}
% question here
\hfill [5 marks]
\begin{flushright}
	[2 marks for correct alternative (T/F); 3 marks for correct justification]\\
\end{flushright}

\hrulefill

\textbf{Answer.} 
\begin{enumerate}[label = (\alph*), nosep] 
	\item F \hfill \textbf{[1]}
	\item F \hfill \textbf{[1]}\\
\end{enumerate}

Justification:
\begin{enumerate}[label = (\alph*)] 
	\item To see this, note that the sequence $(x_n, y_n) = (1/n, 0) \in \mathbb{R}^2$ converges to $(0,0),$ but $f(x_n, y_n) = 1$ for all $n \in \mathbb{N}.$ Thus, $f(x_n, y_n) \to 1 \neq 0 = f(0, 0).$\\
	Hence, $f$ is not continuous at $(0, 0).$ \hfill \textbf{[1]}\\~\\
	Note that the sequence $(x_n, y_n) = (1/n, 1/n^2)$ also confirms this. Marks have obviously been awarded for this and any other valid sequence.
	\item We show that $f_x$ does not exist at $(0, 0).$ \hfill \textbf{[1]}\\
	Note that for $h \neq 0,$ we have it that $\dfrac{f(0 + h, 0) - f(0, 0)}{h} = \dfrac{1}{h}.$\\
	Recall the definition of $f_x$ to see that it does not exist since $\displaystyle\lim_{h\to 0}\dfrac{1}{h}$ does not exist. \hfill \textbf{[1]}
\end{enumerate}
\hrulefill

\vspace{0.2 cm}

Points to be noted -
\begin{enumerate} 
	\item Many students have taken just one sequence $(x_n, y_n) \in \mathbb{R}^2$ that converges to $(0,0)$ such that $f(x_n, y_n)$ converges to $0$ and concluded that the function is continuous.\\
	This is \textbf{obviously wrong} as the definition clearly states that the above must happen for \emph{every} sequence with the required property. \\
	No marks have been awarded for the first part in this case. \\
	It is extremely disappointing to see this mistake being made at this stage in the course. \\
	I mean, if checking one sequence were enough, then why even bother to take $(1/n, 1/n)?$ Just take $(x_n, y_n) = (0, 0)$ for all $n$ and arrive at the same conclusion!
	\item For some bizarre reason, many of you have forgotten to write $\displaystyle\lim_{h\to 0}$ when writing the definition of partial derivative. Half a mark has been deducted for that.
	\item Many students have simply differentiated the given rational function with respect to $x$ and said that the derivative exists everywhere without bothering to check the well-defined-ness of the expression at $(0,0).$ Indeed, the expression written is not defined at $(0,0)$ as it has $(x^4 + y^2)^2$ in the denominator.\\
	No marks have been awarded in this case.
	\item For those who have differentiated as mentioned in the above point and then said that the partial derivative (/derivatives) does (/do) not exist since the limit (/limits) does (/do) not exist, no marks have been given for justification.\\
	It has been mentioned many times that the derivative of a function need not be continuous. We have looked at the expression $x^2\sin\left(\dfrac{1}{x}\right)$ numerous times to drive the point home.
	\item Note that it is not necessary to calculate $f_y(0,0)$ and show its existence. We are done as soon as we get that $f_x(0,0)$ does not exist.
	\item Many have concluded the second part by saying ``Both the partial derivatives of $f$ don't exist.''\\
	Note that that is not the correct statement to write. One must instead write ``Not both the partial derivatives exist.'' (One student did indeed write that.)\\
	However, no marks have been deducted for this.
\end{enumerate}
\end{document}