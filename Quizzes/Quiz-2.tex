\documentclass{article}
\usepackage{amsmath, amssymb, amsfonts, amsthm, mathtools}
\usepackage[utf8]{inputenc}
\usepackage[inline]{enumitem}
\usepackage{cancel}
\usepackage{soul}
\usepackage{hyperref}
\newtheorem{theorem}{Theorem}
\newtheorem{lem}{Lemma}
\newtheorem{defn}{Definition}

\setlength\parindent{0pt}

\usepackage{xcolor}
\definecolor{mybgcolor}{RGB}{46, 51, 63} %46, 51, 63

\usepackage{pagecolor}
\pagecolor{mybgcolor}
\color{white}

\usepackage{geometry}
\geometry{
	a4paper,
	total={170mm,257mm},
	left=20mm,
	top=20mm,
}

\title{Short Quiz 2: Solution}
\author{Aryaman Maithani}
\date{14th August, 2019}

\begin{document}
\maketitle

\hrulefill

\textbf{Question.} State whether the following statement is true or false. Justify your answer.\\
Suppose $f:\mathbb{R}\to\mathbb{R}$ is defined by
\[f(x) = x^3 + \sin x + \frac{x}{1 + x^2} \quad \text{for }x \in \mathbb{R}.\]
Then there exists a real number $c$ such that $f(c) = 2019.$ \hfill [5 marks]
\begin{flushright}
	[2 marks for correct alternative (T/F); 3 marks for correct justification]\\~\\
\end{flushright}
\textbf{Answer.} T \hfill \textbf{[2]}\\
Justification: The function $f$ is continuous as it is the sum of $x^3,$ $\sin x,$ and $\frac{x}{1+x^2},$ each of which is a continuous function. The function $\frac{x}{1 + x^2}$ is continuous as it the quotient of two continuous functions where the denominator is never $0.$ \hfill \textbf{[1]}\\~\\
%
There exist $a,\;b\in\mathbb{R}$ such that $f(a) < 2019$ and $f(b) > 2019.$ For example, one can take $a = 0$ and $b = 2019.$ \hfill \textbf{[1]}\\~\\
%
The conclusion follows by applying the Intermediate value theorem on the \emph{interval} $[a, b] \subset \mathbb{R}.$ \hfill \textbf{[1]}\\


\hrulefill

\vspace{0.2 cm}

Points to be noted -
\begin{enumerate} 
	\item For those who have argued only using the justification that $\displaystyle\lim_{x\to \infty}f(x) = \infty$ or $\displaystyle\lim_{x\to -\infty}=-\infty,$ half a mark has been deducted. The reason for this is simply - We have talked only about real numbers when talking about limits of functions. $\pm\infty$ are not real numbers.
	\item For those who have not mentioned the crucial point that IVT can be used since the function is continuous \emph{on an interval,} half a mark has been deducted. This is crucial as functions which are continuous on sets which are not intervals need not have the Intermediate value property.
	\item The choice of $a$ and $b$ is obviously not unique and marks have been awarded as long as the bounds are correct, whether it be as narrow as $[12, 13]$ or as wide as $[0, 10^6].$
	\item No marks have been cut even if the person has not explicitly mentioned the reason for continuity of $\frac{x}{1 + x^2}.$
\end{enumerate}

\end{document}