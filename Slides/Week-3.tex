\documentclass[handout, aspectratio=169]{beamer}
\mode<presentation>{}
\usepackage[utf8]{inputenc}
\newcommand{\fl}[1]{\left\lfloor #1 \right\rfloor}


\title{MA 105 : Calculus\\ D1 - T5, Tutorial 03}  % change
\author{Aryaman Maithani}
\date[14-08-2019]{14th August, 2019}               % change
\institute[IITB]{IIT Bombay}
\usetheme{Warsaw}
\usecolortheme{beetle}
\newtheorem{defn}{Definition}
\newcommand{\cosec}{\operatorname{cosec}}
\begin{document}
\begin{frame}
	\titlepage
\end{frame}
\begin{frame}{Sheet 2}                            % change
	6. Given: $|f(x+h) - f(x)| \le C|h|^\alpha$ for all $x, x + h \in (a, b).$\\
	Assuming $h \neq 0,$ we can write:\\~\\
	$\left|\dfrac{f(x + h) - f(x)}{h}\right| \le C|h|^{\alpha-1}$\\~\\
	$\implies -C|h|^{\alpha-1} \le \dfrac{f(x + h) - f(x)}{h} \le C|h|^{\alpha-1}$\\~\\
	As $\alpha > 1,$ we have it that $\displaystyle\lim_{h\to 0}C|h|^{\alpha-1} = 0.$ Thus, by Sandwich Theorem, we have it that the limit $\displaystyle\lim_{h\to 0}\dfrac{f(x + h) - f(x)}{h}$ exists and is equal to 0. Thus, the function is differentiable, by definition.\\~\\
	By the definition of $f'(x),$ we also have it that $f'(x) = 0.$
\end{frame}
\begin{frame}{Sheet 2}
7. \[\lim_{h\to 0^+}\dfrac{f(c + h) - f(c - h)}{2h}%
		= \lim_{h\to 0^+}\dfrac{f(c + h) - f(c) + f(c) - f(c - h)}{2h}\]

Now, it is given that $f$ is differentiable at $c.$ This means that $\displaystyle\lim_{h\to 0^+}\dfrac{f(c + h) - f(c)}{h}$ exists. Moreover, it is equal to $f'(c).$\\
Similarly, the limit $\displaystyle\lim_{h\to 0^+}\dfrac{f(c) - f(c-h)}{h}$ exists and equals $f'(c).$ Now that we know the existence of these limits, we can split the sum above.\\~\\
$\displaystyle\lim_{h\to 0^+}\dfrac{f(c + h) - f(c) + f(c) - f(c - h)}{2h}$\\~\\
$=\displaystyle\dfrac{1}{2}\left(\lim_{h\to 0^+}\dfrac{f(c + h) - f(c)}{h} + \lim_{h\to 0^+}\dfrac{f(c) - f(c - h)}{h}\right)$\\~\\
$=\dfrac{1}{2}\left(f'(c) + f'(c)\right) = f'(c).$ \hfill $\blacksquare$
\end{frame}
\begin{frame}{Sheet 2}
	(Converse.)\\
	The converse need not be true. That is,
	\[\lim_{h\to 0^+}\dfrac{f(c + h) - f(c - h)}{2h}\]
	may exist but $f$ could still be non-differentiable at $c.$\\
	Show this explicitly using $f(x) := |x|$ as an example.
\end{frame}
\begin{frame}{Sheet 2}
	8. Given: $f(x + y) = f(x)f(y)$ for all $x, y \in \mathbb{R}.$ \hfill (1)\\
	Let $x = y = 0.$ This gives us that $f(0) = \left(f(0)\right)^2.$\\
	Thus, $f(0) = 0$ or $f(0) = 1.$\\~\\
	Case 1. $f(0) = 0.$\\
	Substitute $y = 0$ in (1). Thus, $f(x) = f(0)f(x) = 0.$\\
	Therefore, $f$ is identically $0$ which means it's differentiable everywhere with derivative $0.$ \\
	Verify that $f'(c) = f'(0)f(c)$ does hold for all $x \in \mathbb{R}.$ (We did not need to use the fact that $f$ is differentiable at $0,$ it followed from definition.)\\~\\
	Case 2. $f(0) = 1.$\\
	As $f$ is differentiable at $0,$ we know that:\\
	$\displaystyle\lim_{h\to 0}\dfrac{f(0+h) - f(0)}{h} = f'(0) \implies \displaystyle\lim_{h\to 0}\dfrac{f(h) - 1}{h} = f'(0).$ \hfill (2)\\
\end{frame}
\begin{frame}{Sheet 2}
	Now, let us show that $f$ is differentiable everywhere.\\
	Let $c \in \mathbb{R}.$ We must show that the following limit exists:\\
	$\displaystyle\lim_{h\to 0}\dfrac{f(c + h) - f(c)}{h}$\\~\\
	Using (1), we can write the above expression as:\\
	$\displaystyle\lim_{h\to 0}\dfrac{f(c)f(h) - f(c)}{h} = \lim_{h\to 0}\dfrac{f(c)(f(h) - 1)}{h} = f(c)\cdot\lim_{h\to 0}\dfrac{f(h) - 1}{h}.$\\~\\
	By (2), we know that the above limit exists. Thus, we have it that $f$ is differentiable at $c$ for every $c \in \mathbb{R}.$ Moreover, $f'(c) = f'(0)f(c).$\\~\\
	%
	\textbf{(Optional)} We have gotten that the derivative of $f$ is a scalar multiple of $f.$ Use this to conclude.	
\end{frame}
\begin{frame}{Sheet 2}
	9. (i) Let $f(x) := \cos x$ for $x \in (0, \pi).$ Then $f$ is one-one and continuous. Consider $c \in (0, \pi).$ Now $f'(c) = -\sin c \neq 0.$\\
	Further, $f\left((0, \pi)\right) = (-1, 1).$ If $d \in (-1, 1)$ and $f(c) = \cos c = d,$ then
	\[(f^{-1})'(d) = \dfrac{1}{f'(c)} = -\dfrac{1}{\sin c} = - \dfrac{1}{\sqrt{1 - \cos^2c}} = - \dfrac{1}{\sqrt{1 - d^2}}.\]
	(ii) Let $f(x) := \cosec x$ for $x \in \left(-\dfrac{\pi}{2}, \dfrac{\pi}{2}\right)\setminus\{0\}.$ Then $f$ is one-one and continuous. Consider $c \in \left(-\dfrac{\pi}{2}, \dfrac{\pi}{2}\right)\setminus\{0\}.$ Now $f'(c) = -\cosec c\cot c = -\cosec^2 c\cos c \neq 0.$\\
	Further, $f\left(\left(-\dfrac{\pi}{2}, \dfrac{\pi}{2}\right)\setminus\{0\}\right) = (-\infty, -1)\cup(1, \infty).$ If $|d| > 1$ and $f(c) = \cosec c= d,$ then
	\[(f^{-1})'(d) = \dfrac{1}{f'(c)} = - \dfrac{1}{\cosec^2 c \cos c} = - \dfrac{1}{\cosec^2 c \sqrt{1 - \frac{1}{\cosec^2 c}}} = - \dfrac{1}{|d|\sqrt{d^2 - 1}}.\]
\end{frame}
\begin{frame}{Sheet 2}
	10. Define $g(x) := \dfrac{2x - 1}{x + 1}$ for $x \in \mathbb{R}\setminus\{1\}.$\\~\\
	Given, $y = (f\circ g)(x).$ As $g$ is differentiable in its domain and so is $f,$ we know that $f\circ g$ is differentiable wherever defined and its derivative is given by:
	\[\dfrac{dy}{dx} = (f\circ g)'(x) = f'(g(x))g'(x) = \sin\left((g(x))^2\right)g'(x).\]
	Let us compute $g'(x).$
	$g(x) = \dfrac{2x - 1}{x + 1} = \dfrac{2x + 2 - 3}{x + 1} = 2 - \dfrac{3}{x+1}.$\\~\\
	Using quotient rule, we get that $g'(x) = \dfrac{3}{(x+1)^2}.$
	\[\therefore \dfrac{dy}{dx} = \sin\left(\left(\dfrac{2x - 1}{x+1}\right)^2\right)\dfrac{3}{x+1}\]
\end{frame}
\begin{frame}{Sheet 3}
	2. Assume that the cubic (denote it by $f(x)$) has two roots, $a$ and $b.$ We may assume that $a < b.$ Then, we know the following:\\
	(i) $f$ is continuous on $[a, b],$\\
	(ii) $f$ is differentiable on $(a, b),$ and\\
	(iii) $f(a) = f(b).$\\
	Thus, by Rolle's Theorem, there exists $c \in (a, b)$ such that $f'(c) = 0.$\\
	However, $f'(c) = 3c^2 + p$ cannot be $0$ as $3c^2$ is always non-negative and $p$ is strictly positive.\\
	Note: We have shown that the cubic has \textbf{\emph{at most}} $1$ root. We haven't actually shown that $f$ has \emph{a} root. This can be shown using IVT. (How?)
\end{frame}
\begin{frame}{Sheet 3}
	3. Part 1. We will first show the existence of such an $x_0 \in (a, b).$\\
	\emph{Proof.} $I := [a, b]$ is an interval and $f$ is continuous. Thus, $f$ has the intermediate value property on $I.$ Thus, the range $J := f(I)$ must be an interval. As $f(a)$ and $f(b)$ are of different signs, $0$ lies between them. As $f(a), f(b) \in J$ and $J$ is an interval, we have it that $0 \in J = f(I).$
	Thus, $0 = f(x_0)$ for some $x_0 \in I = (a, b).$ \hfill $\blacksquare$\\~\\
	Part 2. Now we will show the uniqueness of $x_0.$ Assume that there exists $x_1 \in (a, b)$ such that $f(x_1) = 0.$ We may assume that $x_0 < x_1.$\\
	Now, we know the following:\\
	(i) $f$ is continuous on $[x_0, x_1],$\\
	(ii) $f$ is differentiable on $(x_0, x_1),$ and\\
	(iii) $f(x_0) = f(x_1).$\\
	Thus, by Rolle's Theorem, there exists $x_2 \in (x_0, x_1)$ such that $f'(x_2) = 0.$ But this contradicts the hypothesis that $f'(x) \neq 0$ for all $x \in (a, b).$ \hfill $\blacksquare$\\
\end{frame}
\begin{frame}{Lagrange's Mean Value Theorem (MVT)}
	\begin{theorem}[MVT]
		Let $a < b$ and $f:[a, b] \to \mathbb{R}$ be a function such that\\
		(i) $f$ is continuous on $[a, b],$ and\\
		(ii) $f$ is differentiable on $(a, b).$\\
		Then there exists $c \in (a, b)$ such that $f'(c) = \dfrac{f(b) - f(a)}{b - a}.$
	\end{theorem}
\end{frame}
\begin{frame}{Sheet 3}
	5. To prove that $|\sin a - \sin b| \le |a - b|$ for all $a, b \in \mathbb{R}.$\\
	Case 1. $a = b.$ Trivial.\\
	Case 2. $a \neq b.$ Without loss of generality, we can assume that $a < b.$\\
	As $f := \sin$ is continuous and differentiable on $\mathbb{R},$ there exists $c \in (a, b)$ such that $f'(c) = \dfrac{f(b) - f(a)}{b - a}.$ \hfill (By MVT)\\~\\
	Also, we know that $|f'(c)| = |\cos c| \le 1.$\\~\\
	Thus, we have it that $\left|\dfrac{f(b) - f(a)}{b - a}\right| \le 1.$\\~\\
	This is equivalent to what we wanted to prove. \hfill $\blacksquare$
\end{frame}
\begin{frame}{Sheet 3}
	6. Let $c := \dfrac{a+b}{2}.$ It is clear that $a < c < b.$ Moreover, we have it that $2(c - a) = 2(b - c) = b-a.$\\
	By MVT, there exists $c_1 \in (a, c)$ such that $f'(c_1) = \dfrac{f(c) - f(a)}{c - a}$ and there exists $c_2 \in (c, b)$ such that $f'(c_2) = \dfrac{f(b) - f(c)}{b - c}.$ As $c_1$ and $c_2$ belong to disjoint intervals, it is clear that $c_1 \neq c_2.$\\~\\
	Observe that $f'(c_1) + f'(c_2) = \dfrac{f(c) - f(a)}{c - a} + \dfrac{f(b) - f(c)}{b - c} = 2\left(\dfrac{f(c) - f(a) + f(b) - f(c)}{b - a}\right) = 2.$ \hfill $\blacksquare$
\end{frame}
\begin{frame}{Sheet 3}
	8. Assume not. That is, $f(0) \neq 0.$ Then, there are two possibilities.\\
	Case 1. $f(0) > 0.$\\
	The function $f$ satisfies the hypothesis of MVT, thus there must exist $c \in (-a, 0)$ such that $f'(c) = \dfrac{f(0) - f(-a)}{0 - (-a)} = \dfrac{f(0)}{a} + 1.$\\
	As $f(0) > 0$ and $a > 0,$ we get that $f'(c) > 1$ which contradicts the hypothesis.\\~\\
	Case 2. $f(0) < 0.$\\
	The function $f$ satisfies the hypothesis of MVT, thus there must exist $d \in (0, a)$ such that $f'(d) = \dfrac{f(a) - f(0)}{a - 0} =1 - \dfrac{f(0)}{a}.$\\
	As $f(0) < 0$ and $a > 0,$ we get that $f'(d) > 1$ which contradicts the hypothesis.\\~\\
\end{frame}
\begin{frame}{Sheet 3}
	\textbf{(Optional)} Note the following:
	\[\dfrac{f(x) - f(-a)}{x - (-a)} = \dfrac{f(x) - x + x + a}{x + a} = \dfrac{f(x) - x}{x + a} + 1\]
	and
	\[\dfrac{f(a) - f(x)}{a - x} = \dfrac{a - x + x - f(x)}{a - x} = 1 + \dfrac{x - f(x)}{a - x}\]
	Choose $x \in (-a, a)$ and use MVT appropriately to get contradictions for $f(x) > x$ and $f(x) < x.$
\end{frame}
\end{document}