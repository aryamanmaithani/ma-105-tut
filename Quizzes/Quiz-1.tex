\documentclass{article}
\usepackage{amsmath, amssymb, amsfonts, amsthm, mathtools}
\usepackage[utf8]{inputenc}
\usepackage[inline]{enumitem}
\usepackage{cancel}
\usepackage{soul}
\usepackage{hyperref}
\newtheorem{theorem}{Theorem}
\newtheorem{lem}{Lemma}
\newtheorem{defn}{Definition}

\setlength\parindent{0pt}

\usepackage{xcolor}
\definecolor{mybgcolor}{RGB}{46, 51, 63} %46, 51, 63

\usepackage{pagecolor}
\pagecolor{mybgcolor}
\color{white}

\usepackage{geometry}
\geometry{
	a4paper,
	total={170mm,257mm},
	left=20mm,
	top=20mm,
}

\title{Short Quiz 1: Solution}
\author{Aryaman Maithani}
\date{7th August, 2019}

\begin{document}
\maketitle

\hrulefill

\textbf{Question.} State whether the following statement is true or false. Justify your answer.\\
If $S$ is a nonempty subset of $\mathbb{R}$ such that $S$ is bounded above and if $c := \sup S,$ then there exists a sequence $(x_n)$ of elements of $S$ such that $(x_n)$ is convergent and $x_n \to c.$ \hfill [5 marks]
\begin{flushright}
	[2 marks for correct alternative (T/F); 3 marks for correct justification]\\~\\
\end{flushright}
\textbf{Answer.} T\\
Justification: It is given that $c$ is the supremum of $S.$ Thus, any number strictly less than $c$ cannot be an upper bound. Thus, given any $n \in \mathbb{N},$ $c - 1/n$ is not an upper bound. Thus, there exists $x_n \in S$ such that $c-1/n < x_n.$ As $c$ is an upper bound of $S,$ we have it that
\[c - \frac{1}{n} < x_n \le c \quad \forall n \in \mathbb{N}.\]
Thus, by Sandwich Theorem, we can conclude that $x_n \to c.$

\hrulefill

Common mistakes:
\begin{enumerate} 
	\item Many have said that we can take an increasing sequence $(x_n)$ in $S$ and then $x_n \to c.$\\
	This is not true. It is true that such a sequence \emph{will} converge as $S$ is bounded above but it is not necessary that it converges to $c$ as $\sup\{a_n|n\in\mathbb{N}\}$ may not be equal to $c.$\\
	Example: $S = [0, 2]$ and $x_n := 1 - 1/n.$ We have it that $x_n \to 1 \neq 2 = \sup S.$
	%
	\item Many have said that we can take the sequence $x_n = c - 1/n.$ There is no reason that such a sequence will actually be a sequence in $S.$ For example, $S = \{1\}.$\\
	Even if we take the case where $c \not\in S,$ the statement need not be true.\\
	For example, let $S = (0, 1)\setminus\left\{\frac{1}{2},\frac{2}{3},\frac{3}{4},\ldots\right\}.$
	%
	\item Many have tried to construct a counterexample but it does not work in the correct manner. For example, if you pick a specific set $S$ and a specific sequence $(x_n)$ of elements of $S$ such that $x_n \not\to c,$ you have \textbf{not} disproved the question. The question asked for the existence of \emph{\textbf{a}} sequence.\\
	If one did want to give a counterexample, they would have to find a particular set $S$ which satisfies the hypothesis of the question and show that \emph{every} sequence in that set does not converge to $c.$
	%
	\item Similar to the last point, you have \textbf{not} proven the statement either if you choose a specific set $S$ and show the existence of a sequence there. It must work for every $S.$ 
	%
	\item Many have not considered that the constant sequence is a perfectly valid sequence for the case when $c \in S.$
\end{enumerate}
\end{document}