\documentclass[handout, aspectratio=169]{beamer}
\mode<presentation>{}
\usepackage[utf8]{inputenc}
\newcommand{\fl}[1]{\left\lfloor #1 \right\rfloor}


\title{MA 105 : Calculus\\ Simply connected sets}  % change
\author{Aryaman Maithani}
\date[27-10-2019]{27th October, 2019}               % change
\institute[IITB]{IIT Bombay}
\usetheme{Warsaw}
\usecolortheme{beetle}
\newtheorem{defn}{Definition}
\begin{document}
\begin{frame}
	\titlepage
\end{frame}
\begin{frame}{Simply connected sets}
	Let $m \in \mathbb{N}$ and $D \subset \mathbb{R}^m.$\\
	We know that a curve in $D$ is simply a $\mathcal{C}^1$ function $c:[a, b]\to D$ where $a,\;b\in\mathbb{R}$ with $a < b.$\\
	For the purpose of this discussion, we shall assume $a = 0$ and $b = 1.$\\
	We say that $c$ can be \emph{continuously} shrunk to a point $d \in D$ if there is a continuous function $H:[0, 1]\times[0, 1]\to D$ such that
	\begin{enumerate} 
		\item $H(0, t) = c(t)$ for every $t \in [0, 1],$
		\item $H(1, t) = d$ for every $t \in [0, 1],$ and
		\item $H(s, 1) = H(s, 0)$ for every $s \in [0, 1].$
	\end{enumerate}
	This map $H$ is called a homotopy in $D$ between the curve $c$ and the constant curve $d.$\\
	The domain $D$ is said to be simply-connected if $D$ is path-connected and if for every simply closed curve $c$ in $D,$ we have a homotopy $H$ between $c$ and \emph{some} $d \in D.$
\end{frame}
\begin{frame}{Alternate definition}
	The previous definition can also be written in a slightly more concise (but equivalent) way.\\
	Let $D$ and $c$ have the same meaning as before. Moreover, let $S^1 := \{(x, y) \in \mathbb{R}^2 : x^2 + y^2 = 1\}$ and $U^2 := \{(x, y) \in \mathbb{R}^2 : x^2 + y^2 \le 1\}.$\\~\\
	%
	We say that $D$ is simply-connected if $D$ is path-connected and any loop in $D$ defined by $f:S^1 \to D$ can be contracted to a point: there exists a continuous map $F:U^2 \to D$ such that $F$ restricted to $S^1$ is $f.$
\end{frame}
\end{document}