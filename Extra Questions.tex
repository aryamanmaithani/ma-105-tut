\documentclass{article}
\usepackage{amsmath, amssymb, amsfonts, amsthm, mathtools}
\usepackage[utf8]{inputenc}
\usepackage[inline]{enumitem}
\usepackage{cancel}
\usepackage{soul}
\usepackage{hyperref}
\usepackage{centernot}

\setlength\parindent{0pt}

\usepackage{xcolor}
\definecolor{mybgcolor}{RGB}{50, 50, 50} %46, 51, 63

\usepackage{pagecolor}
\pagecolor{mybgcolor}
\color{white}

\usepackage{titlesec}
\titleformat{\section}[block]
  {\normalfont\scshape}{}{0.25cm}{\large}

\usepackage{geometry}
\geometry{
	a4paper,
	total={170mm,257mm},
	left=20mm,
	top=20mm,
}

\title{Extra Questions for MA 105}
\author{Aryaman Maithani}
\date{Semester: Autumn 2019\\ Latest update: \today}

\begin{document}
\maketitle
\hrulefill

Notation:\\
$\mathbb{N} = \{1,\; 2,\; \ldots\}$ denotes the set of natural numbers.\\
$\mathbb{Z} = \mathbb{N} \cup \{0\} \cup \{-n : n\in\mathbb{N}\}$ denotes the set of integers.\\
$\mathbb{Q}$ denotes the set of rational numbers.\\
$\mathbb{R}$ denotes the set of real numbers.
\section{Week 1}
\begin{enumerate}
	\item Let $f$ be any bijection from $\mathbb{N}$ to $\mathbb{Q} \cap [0,\; 1].$\\
	Define the sequence $(a_n)$ of real numbers as: $a_n := f(n) \quad \forall n \in \mathbb{N}.$\\
	Prove that $(a_n)$ diverges or find an example of $f$ such that $(a_n)$ converges.
	%
	\item Let $(a_n)$ be a sequence of real numbers. We say that $(a_n)$ is \emph{slack-convergent} if there is an $a \in \mathbb{R}$ such that the following condition holds.\\
	For every $\epsilon > 0,$ there is $n_0 \in \mathbb{N}$ such that $|a_n - a| \le \epsilon$ for all $n \ge n_0.$\\
	Prove or disprove that a sequence is convergent (in the normal sense) $\iff$ it is slack-convergent.\\~\\
	\textbf{(Additional)} What happens if we change $n \ge n_0$ to $n > n_0?$
	%
	\item Let $(a_n)$ be a sequence of real numbers. We say that $(a_n)$ is \emph{reciprocal-convergent} if there is an $a \in \mathbb{R}$ such that the following condition holds.\\
	For every $\epsilon > 0,$ there is $n_0 \in \mathbb{N}$ such that $|a_n - a| < 1/\epsilon$ for all $n \ge n_0.$\\
	Prove or disprove that a sequence is convergent (in the normal sense) $\iff$ it is reciprocal-convergent.
	%
	\item Let $(a_n)$ be a sequence of real numbers. We say that $(a_n)$ is \emph{natural-convergent} if the following condition holds.\\
	For every $k \in \mathbb{N},$ $\displaystyle\lim_{n\to \infty}|a_{n+k} - a_n| = 0.$\\
	Prove or disprove that a sequence is convergent (in the normal sense) $\iff$ it is natural-convergent.
	%
	\item Let $(a_n)$ be a sequence of real numbers. We say that $(a_n)$ is \emph{weirdly-convergent} if there is an $a \in \mathbb{R}$ such that the following condition holds.\\
	For every $\epsilon > 0,$ there is $n_0 \in \mathbb{N}$ such that $|a_n - a| < \epsilon$ for infinitely many $n \ge n_0.$\\
	Prove or disprove that a sequence is convergent (in the normal sense) $\iff$ it is weirdly-convergent.
	%
	\item Let $(a_n)$ be a sequence of real numbers. We say that $(a_n)$ is \emph{reverse-convergent} if there is an $a \in \mathbb{R}$ such that the following condition holds.\\
	For every $n_0 \in \mathbb{N},$ there is $\epsilon > 0$ such that $|a_n - a| < \epsilon$ for all $n \ge n_0.$\\
	Prove or disprove that a sequence is convergent (in the normal sense) $\iff$ it is reverse-convergent.
	%
	\item Let $S$ be a nonempty subset of $\mathbb{R}$ which is bounded above. Let $(a_n)$ be an increasing sequence in $S$ such that $\displaystyle\lim_{n\to \infty}a_n = L \not\in S.$\\
	 Prove or disprove that $L = \sup S.$ 
\end{enumerate}
For the question(s) in which the implication does not hold in both directions, does it hold in any? If yes, which?
\section{Week 2}
\begin{enumerate}
	\item Show that $f:\mathbb{N}\to\mathbb{R}$ is continuous for any $f.$
	%
	\item Let $f:\mathbb{Q} \to \mathbb{R}$ be a continuous function such that the image (range) of $f$ is a subset of $\mathbb{Q}.$ Let $a,\;b,\;r\in \mathbb{Q}$ be such that $a < b$ and $f(a) < r < f(b).$ Show (with the help of an example) that it is not necessary that there exists some $c \in \mathbb{Q}\cap[a, b]$ such that $f(c) = r.$
	%
	\item Let $f:\mathbb{R} \to \mathbb{R}$ and $c\in\mathbb{R}.$ We say that $f$ is \emph{reverse continuous} at $c$ if for all $\delta > 0,$ there exists $\epsilon > 0$ such that $|x - c| < \delta \implies |f(x) - f(c)| < \epsilon.$\\
	Is this notion of continuity the same as the normal notion?\\
	If not, then give an example of a function which is reverse continuous at a point but not continuous or vice-versa.
	%
	\item  Let $f:\mathbb{R} \to \mathbb{R}$ and $c\in\mathbb{R}.$ We say that $f$ is \emph{upper continuous} at $c$ if for all $\epsilon > 0,$ there exists $\delta > 0$ such that $|x - c| < \delta \implies f(c) \le f(x) < f(c) + \epsilon.$
	\begin{enumerate} 
		\item Prove that a function is continuous at a point if it is upper continuous at that point.
		\item Show that the converse may not be true.
		\item Give an example of a function that is upper continuous at only one point.
		\item Given any $n \in \mathbb{N},$ show that there exists a function that is upper continuous at exactly $n$ points.
		\item Show that there exists a function that is upper continuous at infinitely many points.
		\item Give an example of a function $f$ that is upper continuous everywhere.
		\item Can you give an example of another function $g$ such that $g$ is upper continuous everywhere but $f-g$ is not constant?
	\end{enumerate}
	%
	\item Let $A, B \subset \mathbb{R}$ and $f:A\to B$ be a bijection. Show with the help of an example that $f$ is continuous $\centernot\implies$ $f^{-1}$ is continuous. 
	%
	\item Show that there exists a bijection from $(0, 1)$ to $[0, 1].$
	%
	\item Show that there exists no continuous bijection from $(0, 1)$ to $[0, 1]$ or from $[0, 1]$ to $(0, 1).$
	%
	\item Let $f:A\to B$ be a continuous surjective function. Show that it is possible for $A$ to be a bounded open interval and $B$ to be a bounded closed interval.\\
	Is it possible for $A$ to be a bounded closed interval and $B$ to be a bounded open interval?
	%
	\item Let $f:\mathbb{R}\to\mathbb{R}$ be a function with the intermediate value property. Is it necessary that $f$ is continuous \emph{somewhere}?
	%
	\item Let $f:\mathbb{R}\to\mathbb{R}$ be a function such that given any $c \in \mathbb{R},$ the limit $\displaystyle\lim_{x\to c}f(x)$ exists. Is it necessary that $f$ is continuous \emph{somewhere}?
\end{enumerate}
The last two questions are just for one to think about. I do not expect solutions for those.
\section{Week 3}
\begin{enumerate} 
	\item Let $f:\mathbb{R}\to\mathbb{R}$ be a differentiable function. Let $c \in \mathbb{R}.$ Is it necessary that there exist $a,\;b \in \mathbb{R}$ such that $a < c < b$ and $f'(c) = \dfrac{f(b) - f(a)}{b - a}?$
	%
	\item Let $k\in \mathbb{N}.$ Construct a function $f:\mathbb{R}\to\mathbb{R}$ that is $k$ times differentiable everywhere but not $(k+1)$ times differentiable somewhere.
	%
	\item Construct a function $f:\mathbb{R}\to\mathbb{R}$ which is differentiable at only one point.
	%
	\item Let $f:\mathbb{R}\to\mathbb{R}$ be differentiable. Suppose there is $\alpha \in \mathbb{R}$ such that for all $x \in \mathbb{R},$ $|f'(x)| \le \alpha < 1.$ Let $a_1 \in \mathbb{R}$ and set $a_{n+1} := f(a_n)$ for all $n \in \mathbb{N}.$ Show that the sequence $(a_n)$ converges.
	%
	\item Let $D \subset \mathbb{R}.$ A function $f:D\to \mathbb{R}$ is said to be \emph{convex} if
	\[f(\lambda x + (1 - \lambda)y) \le \lambda f(x) + (1 - \lambda)f(y) \quad \forall x,\;y\in D,\;\forall \lambda\in [0, 1].\]
	
	Prove that if $I$ is an open interval and $f:I\to\mathbb{R}$ is convex, then $f$ is continuous. Where did you use that $I$ is an open interval?\\
	Give an example to show that if $J$ is not an open interval, then a convex function $f:J\to\mathbb{R}$ need not be continuous.
	%
	\item Let $D \subset \mathbb{R}$ and $f:D\to\mathbb{R}$ be a differentiable function. Show by example that $f'(x) = 0 \quad \forall x \in D$ does not imply that $f$ is constant.
	%
	\item Let $D \subset \mathbb{R}$ and $f:D\to\mathbb{R}$ be a differentiable function.\\
	We say that $f$ is increasing if $\forall x,\;y \in D:$ $x \le y \implies f(x) \le f(y).$\\
	Show by example that $f'(x) \ge 0 \quad \forall x\in D$ does not imply that $f$ is increasing.
	%
	\item Show that the implication in the last two questions would be true if $D$ were an interval.
	%
	\item Let $A$ and $B$ be open intervals in $\mathbb{R}$ and $f:A\to B$ be a bijection such that $f$ is differentiable. Show that it is not necessary that $f^{-1}$ is differentiable.
	%
	\item * Construct a function $f_1:\mathbb{R}\to\mathbb{R}$ with the following properties or show that no such function exists:\\
	$1.$ $f_1$ is differentiable everywhere except one point $x_1.$\\
	$2.$ Define $f_2 : \mathbb{R}\setminus\{x_1\} \to \mathbb{R}$ as $f_2(x) := $ derivative of $f_1$ at $x.$ This $f_2$ must be differentiable everywhere in its domain except one point $x_2.$\\
	$3.$ Define $f_3 : \mathbb{R}\setminus\{x_1,\;x_2\} \to \mathbb{R}$ as $f_3(x) := $ derivative of $f_2$ at $x.$ This $f_3$ must be differentiable everywhere in its domain except one point $x_3.$\\
	\vdots

	$n.$ Define $f_n : \mathbb{R}\setminus\{x_1, \cdots, x_{n-1}\} \to \mathbb{R}$ as $f_n(x) := $ derivative of $f_{n-1}$ at $x.$ This $f_n$ must be differentiable everywhere in its domain except one point $x_n.$\\
	\vdots

	(Note that we do not stop at any $n.$) 
\end{enumerate}
\section{Any Week}
\begin{enumerate} 
	\item Let $D \subset \mathbb{R}.$ We say a function $f : D \to \mathbb{R}$ is \emph{uniformly continuous} if for all $\epsilon > 0,$ there exists $\delta > 0$ such that whenever $x,\;y \in D$ and $|x - y| < \delta,$ then $|f(x) - f(y)| < \epsilon.$
	\begin{enumerate} 
		\item Understand how this definition is different from the definition of (usual) continuity.
		\item Give an example of a function which is continuous but not uniformly continuous.
		\item Show that any uniformly continuous function is also continuous.
	\end{enumerate}
	\item Let $(f_n)$ be a sequence of real valued functions defined on $[a, b]$ such that each $f_n$ is continuous. Moreover, you are given that for each $x \in [a, b],$ the limit $\displaystyle\lim_{n\to \infty}f_n(x)$ exists. \\
	Define the function $f : [a, b] \to \mathbb{R}$ as follows:
	\[f(x) := \lim_{n\to \infty}f_n(x).\]
	Show with the help of an example that it is not necessary that $f$ is continuous.
	\item Let $f_n : D \to \mathbb{R}$ be a sequence of functions from the set $D \subset \mathbb{R}$ to $\mathbb{R}.$ We say that the sequence $(f_n)$ \emph{converges uniformly} to the function $f:D \to \mathbb{R}$ if given $\epsilon > 0,$ there exists an integer $N$ such that
	\[|f_n(x) - f(x)| < \epsilon\]
	for all $n > N$ and all $x \in D.$\\
	Prove that if $(f_n)$ is a sequence of continuous functions that converges uniformly to $f,$ then $f$ is continuous.\\
	If you have solved the previous question, show that $(f_n)$ didn't uniformly converge to $f$ for that example.
	\item Let $f:[a, b] \to \mathbb{R}$ be any function. Then, we know that if
	\begin{enumerate} 
		\item $f$ is monotonic, or
		\item $f$ is bounded and has at most a finite number of discontinuities in $[a, b],$
	\end{enumerate}
	then $f$ is (Riemann) integrable.\\
	Is the converse true?\\
	That is, if $f$ is (Riemann) integrable, then is it necessary that one of (a) or (b) should be true? Prove or disprove via counterexample. \hfill (Credit: Amit)
	%
	\item Show that any function $f:\mathbb{N} \to \mathbb{R}$ is uniformly continuous.
	\item Let $a \in \mathbb{R}$ and $(a_n)$ be a sequence of real numbers with the following property: Given any subsequence $\left(a_{n_k}\right)$ of $(a_n),$ there exists a subsequence $\left(a_{n_{k_l}}\right)$ of $\left(a_{n_k}\right)$ with the property that $\displaystyle\lim_{l\to \infty}a_{n_{k_l}}	= a.$\\
	Prove that $\displaystyle\lim_{n\to \infty}a_n = a.$
	\item Let $E$ be a bounded subset of $\mathbb{R}$ with the following property:\\
	There exists $x_0 \in \mathbb{R}\setminus E$ such that there exists a sequence $(x_n)$ in $E$ which converges to $x_0.$ (For those familiar with the lingo, $E$ is not a closed set.)\\
	Show that there exists:
	\begin{enumerate}[nosep] 
		\item A function $g:E\to \mathbb{R}$ which is continuous but not bounded.
		\item A function $f:E\to \mathbb{R}$ such that $f(E)$ is bounded but does not have a maximum.
		\item A function $h:E\to \mathbb{R}$ such that $h$ is continuous but not uniformly continuous.
	\end{enumerate}
	\item Let $f:(a, b) \to \mathbb{R}$ be a monotonically increasing function, that is, $a < x < y < b \implies f(x) \le f(y).$\\
	Show that for any $x \in (a, b),$ both $\displaystyle\lim_{t\to x^-}f(t)$ and $\displaystyle\lim_{t\to x^+}f(t)$ exist. Moreover, show that $\displaystyle\lim_{t\to x^-}f(t) \le f(x) \le \displaystyle\lim_{t\to x^+}f(t).$\\~\\
	Also show that if $x < y,$ then $\displaystyle\lim_{t\to x^+}f(t) \le \displaystyle \lim_{t\to y^-}f(t).$\\~\\
	(Hint: Try relating $\displaystyle\lim_{t\to x^-}f(t)$ with $\displaystyle\sup_{a < t < x}f(t).$)
	\item Let $S = \{a + b\sqrt{2} : a, b \in \mathbb{Z}\}.$ Show that given any $x \in \mathbb{R},$ there exists a sequence $(s_n)$ in $S$ that converges to $x.$\\
	Bonus 1: Generalise the argument by replacing $\sqrt{2}$ by any irrational square root of a natural number.\\
	Bonus 2: Generalise the argument by replacing $\sqrt{2}$ by any irrational number.
	\item Let $f:\mathbb{R} \to \mathbb{R}$ be a periodic function with period $p > 0.$ That is, $f(x+p) = f(x)$ for all $x \in \mathbb{R}.$ Moreover, assume that $f$ is Riemann integrable on $[x, x+p]$ for any $x \in \mathbb{R}.$ Is it necessary that $\displaystyle\int_{x}^{x+p} f(x) dx $ is independent of $x?$ (Note that $f$ is not necessarily continuous.)
	\item Let $A \subset \mathbb{R}$ and $f:A\to\mathbb{R}$ be a continuous and periodic function.
	\begin{enumerate}[nosep] 
		\item Show that if $A = \mathbb{R},$ then $f$ is bounded.
		\item Show that there exists some $A$ and some $f$ for which the hypothesis holds but $f$ is not bounded.
	\end{enumerate}
\end{enumerate}
\end{document}