\documentclass{article}
\usepackage{amsmath, amssymb, amsfonts, amsthm, mathtools}
\usepackage[utf8]{inputenc}
\usepackage[inline]{enumitem}
\usepackage{cancel}
\usepackage{soul}
\usepackage{hyperref}
\usepackage{centernot}

\setlength\parindent{0pt}

\usepackage{xcolor}
\definecolor{mybgcolor}{RGB}{50, 50, 50} %46, 51, 63

\usepackage{pagecolor}
\pagecolor{mybgcolor}
\color{white}

\usepackage{titlesec}
\titleformat{\section}[block]
  {\normalfont\scshape}{}{0.25cm}{\large}

\usepackage{geometry}
\geometry{
	a4paper,
	total={170mm,257mm},
	left=20mm,
	top=20mm,
}

\title{Extra Questions for MA 105}
\author{Aryaman Maithani}
\date{Semester: Autumn 2019}

\begin{document}
\maketitle
\hrulefill

Notation:\\
$\mathbb{N} = \{1,\; 2,\; \ldots\}$ denotes the set of natural numbers.\\
$\mathbb{Q}$ denotes the set of rational numbers.\\
$\mathbb{R}$ denotes the set of real numbers.
\section{Week 1}
\begin{enumerate}
	\item Let $f$ be any bijection from $\mathbb{N}$ to $\mathbb{Q} \cap [0,\; 1].$\\
	Define the sequence $(a_n)$ of real numbers as: $a_n := f(n) \quad \forall n \in \mathbb{N}.$\\
	Prove that $(a_n)$ diverges or find an example of $f$ such that $(a_n)$ converges.
	%
	\item Let $(a_n)$ be a sequence of real numbers. We say that $(a_n)$ is \emph{slack-convergent} if there is an $a \in \mathbb{R}$ such that the following condition holds.\\
	For every $\epsilon > 0,$ there is $n_0 \in \mathbb{N}$ such that $|a_n - a| \le \epsilon$ for all $n \ge n_0.$\\
	Prove or disprove that a sequence is convergent (in the normal sense) $\iff$ it is slack-convergent.\\~\\
	\textbf{(Additional)} What happens if we change $n \ge n_0$ to $n > n_0?$
	%
	\item Let $(a_n)$ be a sequence of real numbers. We say that $(a_n)$ is \emph{reciprocal-convergent} if there is an $a \in \mathbb{R}$ such that the following condition holds.\\
	For every $\epsilon > 0,$ there is $n_0 \in \mathbb{N}$ such that $|a_n - a| < 1/\epsilon$ for all $n \ge n_0.$\\
	Prove or disprove that a sequence is convergent (in the normal sense) $\iff$ it is reciprocal-convergent.
	%
	\item Let $(a_n)$ be a sequence of real numbers. We say that $(a_n)$ is \emph{natural-convergent} if the following condition holds.\\
	For every $k \in \mathbb{N},$ $\displaystyle\lim_{n\to \infty}|a_{n+k} - a_n| = 0.$\\
	Prove or disprove that a sequence is convergent (in the normal sense) $\iff$ it is natural-convergent.
	%
	\item Let $(a_n)$ be a sequence of real numbers. We say that $(a_n)$ is \emph{weirdly-convergent} if there is an $a \in \mathbb{R}$ such that the following condition holds.\\
	For every $\epsilon > 0,$ there is $n_0 \in \mathbb{N}$ such that $|a_n - a| < \epsilon$ for infinitely many $n \ge n_0.$\\
	Prove or disprove that a sequence is convergent (in the normal sense) $\iff$ it is weirdly-convergent.
	%
	\item Let $(a_n)$ be a sequence of real numbers. We say that $(a_n)$ is \emph{reverse-convergent} if there is an $a \in \mathbb{R}$ such that the following condition holds.\\
	For every $n_0 \in \mathbb{N},$ there is $\epsilon > 0$ such that $|a_n - a| < \epsilon$ for all $n \ge n_0.$\\
	Prove or disprove that a sequence is convergent (in the normal sense) $\iff$ it is reverse-convergent.
	%
	\item Let $S$ be a nonempty subset of $\mathbb{R}$ which is bounded above. Let $(a_n)$ be an increasing sequence in $S$ such that $\displaystyle\lim_{n\to \infty}a_n = L \not\in S.$\\
	 Prove or disprove that $L = \sup S.$ 
\end{enumerate}
For the question(s) in which the implication does not hold in both directions, does it hold in any? If yes, which?
\section{Week 2}
\begin{enumerate}
	\item Show that $f:\mathbb{N}\to\mathbb{R}$ is continuous for any $f.$
	%
	\item Let $f:\mathbb{Q} \to \mathbb{R}$ be a continuous function such that the image (range) of $f$ is a subset of $\mathbb{Q}.$ Let $a,\;b,\;r\in \mathbb{Q}$ be such that $a < b$ and $f(a) < r < f(b).$ Show (with the help of an example) that it is not necessary that there exists some $c \in \mathbb{Q}\cap[a, b]$ such that $f(c) = r.$
	%
	\item Let $f:\mathbb{R} \to \mathbb{R}$ and $c\in\mathbb{R}.$ We say that $f$ is \emph{reverse continuous} at $c$ if for all $\delta > 0,$ there exists $\epsilon > 0$ such that $|x - c| < \delta \implies |f(x) - f(c)| < \epsilon.$\\
	Is this notion of continuity the same as the normal notion?\\
	If not, then give an example of a function which is reverse continuous at a point but not continuous or vice-versa.
	%
	\item  Let $f:\mathbb{R} \to \mathbb{R}$ and $c\in\mathbb{R}.$ We say that $f$ is \emph{upper continuous} at $c$ if for all $\epsilon > 0,$ there exists $\delta > 0$ such that $|x - c| < \delta \implies f(c) \le f(x) < f(c) + \epsilon.$
	\begin{enumerate} 
		\item Prove that a function is continuous at a point if it is upper continuous at that point.
		\item Show that the converse may not be true.
		\item Give an example of a function that is upper continuous at only one point.
		\item Given any $n \in \mathbb{N},$ show that there exists a function that is upper continuous at exactly $n$ points.
		\item Show that there exists a function that is upper continuous at infinitely many points.
		\item Give an example of a function $f$ that is upper continuous everywhere.
		\item Can you give an example of another function $g$ such that $g$ is upper continuous everywhere but $f-g$ is not constant?
	\end{enumerate}
	%
	\item Let $A, B \subset \mathbb{R}$ and $f:A\to B$ be a bijection. Show with the help of an example that $f$ is continuous $\centernot\implies$ $f^{-1}$ is continuous. 
	%
	\item Show that there exists a bijection from $(0, 1)$ to $[0, 1].$
	%
	\item Show that there exists no continuous bijection from $(0, 1)$ to $[0, 1]$ or from $[0, 1]$ to $(0, 1).$
	%
	\item Let $f:A\to B$ be a continuous surjective function. Show that it is possible for $A$ to be a bounded open interval and $B$ to be a bounded closed interval.\\
	Is it possible for $A$ to be a bounded closed interval and $B$ to be a bounded open interval?
	%
	\item Let $f:\mathbb{R}\to\mathbb{R}$ be a function with the intermediate value property. Is it necessary that $f$ is continuous \emph{somewhere}?
	%
	\item Let $f:\mathbb{R}\to\mathbb{R}$ be a function such that given any $c \in \mathbb{R},$ the limit $\displaystyle\lim_{x\to c}f(x)$ exists. Is it necessary that $f$ is continuous \emph{somewhere}?
\end{enumerate}
The last two questions are just for one to think about. I do not expect solutions for those.
\end{document}