\documentclass[handout, aspectratio=169]{beamer}
\mode<presentation>{}
\usepackage[utf8]{inputenc}
\newcommand{\fl}[1]{\left\lfloor #1 \right\rfloor}


\title{MA 105 : Calculus\\ D1 - T5, Tutorial 10}  % change
\author{Aryaman Maithani}
\date[9-10-2019]{9th October, 2019}               % change
\institute[IITB]{IIT Bombay}
\usetheme{Warsaw}
\usecolortheme{beetle}
\newtheorem{defn}{Definition}
\begin{document}
\begin{frame}
	\titlepage
\end{frame}
\begin{frame}{Summary} 
	Sheet 7: Problems 5, 6, 7, 8, 9, 10
\end{frame}
\begin{frame}{Sheet 7}                            % change
	5. (i) $f(x, y) = x^4 + y^4 + 4x - 32y -7, \quad (x_0, y_0) = (-1, 2).$\\
	\uncover<1->{Note that the above function is defined on $D = \mathbb{R}^2.$}\\
	\uncover<2->{Thus, the given point is an interior point of $D.$ Moreover, it can be seen that the partial derivatives of all orders exist and are continuous everywhere. }\\
	\uncover<3->{Note that $(\nabla f)(x, y) = (4x^3 + 4, 4y^3 - 32)$.} \uncover<4->{Hence, $(\nabla f)(x_0, y_0) = 0.$ }\\
	\uncover<5->{Thus, we can appeal to the determinant test. }\\~\\
	\uncover<6->{$(\Delta f)(x, y) = (12x^2)(12y^2) - (0)^2 = 144x^2y^2.$ }\\
	\uncover<7->{Thus, $(\Delta f)(x_0, y_0) > 0.$ }\\
	\uncover<8->{Also, $f_{xx}(x_0, y_0) = 12x_0^2 > 0.$ }\\~\\
	\uncover<9->{Thus, by the determinant test, we get that $f$ has a local minimum at $(x_0, y_0).$ }
\end{frame}
\begin{frame}{Sheet 7}
	5. (ii) $f(x, y)=x^{3}+3 x^{2}-2 x y+5 y^{2}-4 y^{3}, \quad\left(x_{0}, y_{0}\right)=(0,0).$\\
	\uncover<1->{Note that the above function is defined on $D = \mathbb{R}^2.$}\\
	\uncover<2->{Thus, the given point is an interior point of $D.$ Moreover, it can be seen that the partial derivatives of all orders exist and are continuous everywhere. }\\
	\uncover<3->{Note that $(\nabla f)(x, y) = (3x^2 + 6x - 2y, -2x + 10y - 12y^2)$.} \uncover<4->{Hence, $(\nabla f)(x_0, y_0) = 0.$ }\\
	\uncover<5->{Thus, we can appeal to the determinant test. }\\~\\
	\uncover<6->{$(\Delta f)(x, y) = (6x+6)(10 - 24y) - (-2)^2.$ }\\
	\uncover<7->{Thus, $(\Delta f)(x_0, y_0) = (6)(10) - 4 = 56 > 0.$ }\\
	\uncover<8->{Also, $f_{xx}(x_0, y_0) = 6 > 0.$ }\\~\\
	\uncover<9->{Thus, by the determinant test, we get that $f$ has a local minimum at $(x_0, y_0).$ }
\end{frame}
\begin{frame}{Sheet 7}
	6. (i) $f(x, y)=\left(x^{2}-y^{2}\right) e^{-\left(x^{2}+y^{2}\right) / 2}.$\\
	\uncover<1->{Note that the above function is defined on $D = \mathbb{R}^2.$}\\
	\uncover<2->{Thus, every point is an interior point of $D.$ Moreover, it can be seen that the partial derivatives of all orders exist and are continuous everywhere. } \hfill \uncover<3->{(How?) }\\
	\uncover<4->{For $(x_0, y_0)$ to be a point of extrema or a saddle point, it must be the case that $(\nabla f)(x_0, y_0) = (0, 0).$ }\\~\\
	\uncover<5->{Note that $f_x(x, y) =x e^{1 / 2\left(-x^{2}-y^{2}\right)}\left(-x^{2}+y^{2}+2\right).$ }\\
	\uncover<6->{Also, $f_y(x, y) =y e^{1 / 2\left(-x^{2}-y^{2}\right)}\left(-x^{2}+y^{2}-2\right).$ }\\~\\
	\uncover<7->{Thus, solving $(\nabla f)(x_0, y_0) = (0, 0)$ gives us that $(x_0, y_0) \in \{(0, 0),\;(0, \sqrt{2}),\;(0, -\sqrt{2}),\;(-\sqrt{2},0),\;(\sqrt{2}, 0)\}.$ }\\
	\uncover<8->{Now, we determine the exact nature using the determinant test. }
\end{frame}
\begin{frame}{Sheet 7}
	Recall that $(\Delta f)\left(x_{0}, y_{0}\right):=f_{x x}\left(x_{0}, y_{0}\right) f_{y y}\left(x_{0}, y_{0}\right)-f_{x y}\left(x_{0}, y_{0}\right)^{2}.$\\
	\uncover<2->{Hence, in our case,
	\[(\Delta f)(x, y) = -e^{-x^{2}-y^{2}}\left(x^{6}-x^{4} y^{2}-3 x^{4}-x^{2} y^{4}+22 x^{2} y^{2}-8 x^{2}+y^{6}-3 y^{4}-8 y^{2}+4\right).\] }\\
	\uncover<3->{Moreover, $f_{xx}(x, y) = e^{-\left(x^{2}+y^{2}\right) / 2}(x^4 - x^2y^2 - 5x^2 + y^2 + 2)$ }\\
	\uncover<3->{For $(x_0, y_0) = (0, 0),$ it is clear that it is a saddle point for $f$ as discriminant is $-4 < 0.$ }\\~\\
	\uncover<4->{Note that if $x = 0,$ the discriminant reduces to $-e^{-y^2}(y^6 - 3y^4 -8y^2 + 4).$ }\\
	\uncover<5->{Substituting $y = \pm\sqrt{2}$ gives us that the discriminant is positive with $f_{xx}$ positive and hence, the points are points of local minima. }\\~\\
	\uncover<6->{Similarly, we get that the points $(\pm\sqrt{2}, 0)$ are points of local maxima as they have discriminant positive and $f_{xx}$ negative. }
\end{frame}
\begin{frame}{Sheet 7}
	6. (ii) $f(x, y)=f(x, y)=x^{3}-3 x y^{2}.$\\
	\uncover<1->{Note that the above function is defined on $D = \mathbb{R}^2.$}\\
	\uncover<2->{Thus, every point is an interior point of $D.$ Moreover, it can be seen that the partial derivatives of all orders exist and are continuous everywhere. } \hfill \uncover<3->{(How?) }\\
	\uncover<4->{For $(x_0, y_0)$ to be a point of extrema or a saddle point, it must be the case that $(\nabla f)(x_0, y_0) = (0, 0).$ }\\~\\
	\uncover<5->{Note that $f_x(x, y) = 3x^2 - 3y^2.$ }\\
	\uncover<6->{Also, $f_y(x, y) = -6xy.$ }\\~\\
	\uncover<7->{Thus, solving $(\nabla f)(x_0, y_0) = (0, 0)$ gives us that $(x_0, y_0) = (0, 0).$ }\\
	\uncover<8->{Now, we determine the exact nature using the determinant test. }
\end{frame}
\begin{frame}{Sheet 7}
	Recall that $(\Delta f)\left(x_{0}, y_{0}\right):=f_{x x}\left(x_{0}, y_{0}\right) f_{y y}\left(x_{0}, y_{0}\right)-f_{x y}\left(x_{0}, y_{0}\right)^{2}.$\\
	\uncover<2->{Hence, in our case,
	\[(\Delta f)(x_0, y_0) = -36(x_0^2 + y_0^2).\] }
	\uncover<3->{Thus, for $(x_0, y_0) = (0, 0),$ we get the discriminant is $0.$}\\
	\uncover<4->{Hence, we get that }\uncover<5->{the discriminant test is {\color[rgb]{1, 0, 0} inconclusive!} }\\
	\uncover<6->{This means that we must turn to some other analytic methods of determining the nature. }\\~\\
	\uncover<7->{Now, we note that $f(\delta, 0) = \delta^3$ for all $\delta \in \mathbb{R}.$ }\\
	\uncover<8->{Thus, given any $\epsilon > 0,$ choose $\delta = \pm \epsilon/2.$ }\\
	\uncover<9->{This gives us that $(0, 0)$ is saddle point. } \hfill \uncover<10->{(How?) }
\end{frame}
\begin{frame}{Sheet 7}
	7. To find: Absolute maxima and minima of $f(x, y)=\left(x^{2}-4 x\right) \cos y \text { for } 1 \leq x \leq 3,-\pi / 4 \leq y \leq \pi / 4.$\\
	Note that the domain is a closed and bounded set. As $f$ is continuous on the domain, $f$ does achieve a maximum and a minimum.
	\uncover<2->{ Note that $f_x(x, y) = (2 x-4) \cos y$ and $f_y(x, y) = -\left(x^{2}-4 x\right) \sin y$ for interior points $(x, y).$}\\
	\uncover<3->{Thus, the only critical point is $p_1 = (2, 0).$}\\~\\
	\uncover<4->{Now we restrict ourselves to the boundaries to find the local extrema.}\\
	\uncover<5->{``Right boundary:'' This is the line segment $x = 3, -\pi / 4 \leq y \leq \pi / 4.$ }\\
	\uncover<6->{The function now reduces to $-3\cos y$ on this segment. }\\
	\uncover<7->{Using our theory from one-variable calculus, we get that we need to check the points $(3, 0),\;(3, \pi/4),\;(3, -\pi/4).$ } \hfill \uncover<7->{(How?)}\\~\\
	\uncover<8->{Similar consideration of the ``left boundary'' gives us the points $(1, 0),\;(1, \pi/4),\;(1, -\pi/4).$}
\end{frame}
\begin{frame}{Sheet 7}
	Now, we look at the ``top boundary.''\\
	\uncover<2->{The function there reduces to $\frac{x^2 - 4x}{\sqrt{2}}.$ }\\
	\uncover<3->{Once again, using our theory from one-variable calculus, we get that we need to check the points $(1, \pi/4),\;(2, \pi/4),\;(3, \pi/4).$ }\\~\\
	\uncover<4->{Similarly, checking the ``bottom boundary'' gives us the points $(1, -\pi/4),\;(2, -\pi/4),\;(3, -\pi/4).$ }\\
	\uncover<5->{We now tabulate our results as follows: }
	\uncover<6->{
	\[\begin{array}{|c||c|c|c|c|c|}
	\hline
	(x_0, y_0) & (2, 0) & (3, 0) & (3, \pi/4) & (2, \pi/4) & (1, \pi/4) \\
	\hline
	f(x_0, y_0) & -4 & -3 & \dfrac{-3}{\sqrt{2}} & \dfrac{-4}{\sqrt{2}} & \dfrac{-3}{\sqrt{2}} \\
	\hline
	\hline
	(x_0, y_0) & (1, 0) & (1, -\pi/4) & (2, -\pi/4) & (3, -\pi/4) &  \\
	\hline
	f(x_0, y_0) & -3 & \dfrac{-3}{\sqrt{2}} & \dfrac{-4}{\sqrt{2}} & \dfrac{-3}{\sqrt{2}} & \\
	\hline 
	\end{array}
	\]
	}
	\uncover<7->{Thus, we get that $f_{\text{min}} = -4$ at $(2, 0)$ and $f_{max} = -\frac{3}{\sqrt{2}}$ at $(1, \pm \pi/4)$ and $(3, \pm\pi/4).$}
\end{frame}
\begin{frame}{Sheet 7}
	8. Let $g:\mathbb{R}^3 \to \mathbb{R}$ be defined as $g(x, y, z) := x^2 + y^2 + z^2 - 1.$ We need to maximise the function $T(x, y, z) = 400xyz$ subject to the constraint $g = 0.$ 
	\uncover<2->{Note that the set $S^2 = \{(x, y, z) \in \mathbb{R}^3 : g(x, y, z) = 0\}$ is nonempty, closed and bounded, and $T$ is continuous on it. Thus, $f$ will attain its maximum on $S^2.$ }\\
	\uncover<3->{Now, $(\nabla T)(x, y, z)=\lambda(\nabla g)(x, y, z) \text { and } g(x, y, z)=0$ means}
	\uncover<4->{\[400 y z = 2 \lambda x, \quad 400 x z = 2 \lambda y, \quad 400 x y = 2 \lambda z, \quad x^2 + y^2 + z^2 = 1.\] }\\
	\uncover<5->{The above gives us that $400 x y z= 2 \lambda x^{2}= 2 \lambda y^{2}= 2 \lambda z^{2}.$ }\\~\\
	\uncover<6->{Also, $(\nabla g)(x, y, z) \neq (0, 0, 0)$ whenever $g(x, y, z) = 0.$ }\\
	\uncover<7->{Thus, the hypotheses of the Lagrange Multiplier Theorem are satisfied. }
\end{frame}
\begin{frame}{Sheet 7}
	Now, we solve the equations to get the points of maxima.\\
	\uncover<2->{If $\lambda \neq 0,$ then $x^2 = y^2 = z^2$ and hence, we get the $8$ points $\left(\pm\frac{1}{\sqrt{3}}, \pm\frac{1}{\sqrt{3}}, \pm\frac{1}{\sqrt{3}}\right).$ }\\
	\uncover<3->{If $\lambda = 0,$ then $yz = zx = xy = 0.$ This, combined with $g = 0$ gives us the $6$ points $(0, 0, \pm 1),\;(0, \pm 1, 0),\;(\pm 1, 0, 0).$ }\\~\\
	\uncover<4->{Now, we check the value of $T$ at these $14$ points.}\\
	\uncover<5->{The first $8$ points give either $T = \frac{400}{3\sqrt{3}}$ or $T = -\frac{400}{3\sqrt{3}}.$ The last $6$ points give $T = 0.$ }\\~\\
	\uncover<6->{Thus, the highest value of $T$ is {\color[rgb]{1, 0, 0}$\dfrac{400}{3\sqrt{3}}.$} }
\end{frame}
\begin{frame}{Sheet 7}
	9. We wish to maximise $f(x, y, z)=x y z$ subject to the constraints $g(x, y, z) = x+y+z - 40 = 0 \text { and } h(x, y, z) = x+y - z = 0.$\\
	$g = h = 0$ clearly gives us that $z = 20.$\\
	Using this and $h,$ we get that $x + y = 20.$\\
	Thus, $f(x, y, z) = 20x(20 - x) = -20\left((x-10)^2 - 100\right).$\\
	Note that $(x-10)^2\ge0$ and hence, $f(x, y, z) \le 2000.$\\
	Thus, we get that $f$ is bounded above by $2000$ under the constraints $g=h=0.$\\
	Moreover, $f$ does attain this value at $(10, 10, 20)$ which is a point satisfying the constraints.\\
	Thus, we get that the maximum value attained by $f$ is $2000,$ given the constraints.\\~\\
	Note that this time, our constraint set was not bounded. Thus, we had no reason to assume a priori that the maximum is attained.
\end{frame}
\begin{frame}{Sheet 7}
	10. We wish to maximise $f(x, y, z)=x^2 + y^2 + z^2$ subject to the constraints $g(x, y, z) = x+2 y+3 z - 6=0 \text { and }  h(x, y, z) = x+3 y+4 z - 9= 0$\\
	Note that the set $E := \{(x, y, z) \in \mathbb{R}^3 : g(x, y, z) = h(x, y, z) = 0\}$ is \textbf{not} bounded. Thus, we can't straight away say that $f$ does indeed attain a minimum on $E.$\\
	However, observe that $(0, 3, 0) \in E$ and $f(0, 3, 0) = 9.$ Thus, if there were to exist a global minimum at some point $(x, y, z),$ then it would have to be the case that $f(x, y, z) \le 9.$ This motivates us to consider the new set
	\[E' = \{(x, y, z) \in E: f(x, y, z) \le 9\}.\]
	This set is clearly bounded. Moreover, it is closed and non-empty as well. Thus, $f$ attains a minimum on $E'$ which would in turn be a minimum on $E$ as well. Now, we turn back to Lagrange.
\end{frame}
	
\begin{frame}{Sheet 7}
	\uncover<2->{We solve $\nabla f = \lambda\nabla g + \mu\nabla h$ along with $g = h = 0$ for $\lambda, \mu, x, y, z.$}\\~\\
	\uncover<3->{We see that $\nabla f = (2x, 2y, 2z),\;\nabla g = (1, 2, 3),\;\nabla h = (1, 3, 4).$ \\
	Thus, it is clear that $\nabla g$ and $\nabla h$ are always non-zero. Moreover, they are non-parallel at all points. }\\
	\uncover<4->{Thus, we get $2x = \lambda + \mu,\;2y = 2\lambda + 3\mu,\;2z = 3\lambda + 4\mu.$ \hfill $(*)$ }\\~\\
	\uncover<5->{Note that $2g = 2h = 0$ along with $(*)$ gives us that $\lambda = -10,\;\mu = 8.$ }\\
	\uncover<6->{Now, the equalities of $(*)$ give us that $x = -1,\;y = 2,\;z = 1.$ It is clear that this is indeed a point of minimum. }\\~\\
\end{frame}
\end{document}