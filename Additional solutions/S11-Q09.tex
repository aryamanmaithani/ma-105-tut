\documentclass[handout, aspectratio=169]{beamer}
\mode<presentation>{}
\usepackage[utf8]{inputenc}
\newcommand{\fl}[1]{\left\lfloor #1 \right\rfloor}


\title{MA 105 : Calculus\\ Sheet 11 Question 9}  % change
\author{Aryaman Maithani}
\date[31-10-2019]{31st October, 2019}               % change
\institute[IITB]{IIT Bombay}
\usetheme{Warsaw}
\usecolortheme{beetle}
\newtheorem{defn}{Definition}
\begin{document}
\begin{frame}{Sheet 11 Question 9}                            % change
	(9) Let us first prove these so-called Green's identities as we can't recall something that we haven't seen before.\\
	All of these proofs are actually quite simple. We shall continuously be using the normal form of Green's theorem, that is,
	\[\int_{\partial R}\mathbf{F}\cdot\mathbf{\hat{n}}ds = \iint_R(\nabla\cdot\mathbf{F})d(x, y).\]
	Also, I shall point that the notation $\dfrac{\partial w}{\partial \mathbf{n}}$ stands for $(\nabla w)\cdot\mathbf{\hat{n}}.$\\~\\
	(i)	Use the above normal form for $\mathbf{F} = \nabla w.$\\
	(ii) Use the above normal form for $\mathbf{F} = \frac{1}{2}\nabla (w^2)$ and use the identities (2)(i) and (3)(i) from Sheet 9.\\
	(iii) Simply consider the normal form for $\mathbf{F} = v\nabla w - w\nabla v$ and use the product identities as before.
\end{frame}
\begin{frame}{Sheet 11 Question 9} 
	(a) We quite simply get $\nabla^2 w = 0$ and hence the desired integral is zero by part (i).\\~\\
	(b) Define $\mathbf{H} = \mathbf{F} - \mathbf{G}.$ Then, $\nabla\times\mathbf{H}=\mathbf{0}.$ Since $D$ is simply connected, there exists a scalar field $\varphi$ on $D$ such that $\nabla \varphi = \mathbf{H}.$ \hfill (Why is $D$ simply connected?)\\
	Observe that $\nabla\cdot\mathbf{H} = 0$ implies that $\nabla^2\varphi = 0.$\\
	Lastly, $\mathbf{H}\cdot\mathbf{N} = 0$ gives us that $(\nabla\varphi)\cdot\mathbf{N} = 0$ on the boundary $C.$ Now using the second identity with $w = \varphi$ gives us that:
	\[\iint_D [(\nabla \varphi)\cdot(\nabla \varphi)]d(x, y) = 0,\]
	or 
	\[\iint_D \|\mathbf{H}\|^2d(x, y) = 0.\]
\end{frame}
\begin{frame}{Sheet 11 Question 9} 
	Now, note that $H$ is a smooth vector field. In particular, it is continuous and hence, so is $\|\mathbf{H}\|^2.$ If the integral of a non-negative \textbf{continuous} function is zero, then we know that the function itself must be identically $0,$ that is, it must be $0$ everywhere in the region.\\
	Thus, we get that $\|\mathbf{H}\|^2 = 0$ on $D$ which is equivalent to saying that $\mathbf{H} = \mathbf{0}$ on $D,$ which in turn is the same as:
	\[\mathbf{F} = \mathbf{G} \quad \text{on } D.\]
	\vspace{1 cm}
	
	Note that the hypothesis of $\|\mathbf{H}\|^2$ being continuous is essential to conclude that it must be $0$ everywhere.
\end{frame}
\end{document}