\documentclass{article}
\usepackage{amsmath, amssymb, amsfonts, amsthm, mathtools}
\usepackage[utf8]{inputenc}
\usepackage[inline]{enumitem}
\usepackage{cancel}
\usepackage{soul}
\usepackage{hyperref}
\newtheorem{theorem}{Theorem}
\newtheorem{lem}{Lemma}
\newtheorem{defn}{Definition}

\setlength\parindent{0pt}

\usepackage{xcolor}
\definecolor{mybgcolor}{RGB}{46, 51, 63} %46, 51, 63

\usepackage{pagecolor}
\pagecolor{mybgcolor}
\color{white}

\usepackage{geometry}
\geometry{
	a4paper,
	total={170mm,257mm},
	left=20mm,
	top=20mm,
}

\title{Short Quiz 2: Solution}      % here
\author{Aryaman Maithani}
\date{4th September, 2019}  		 % here

\begin{document}
\maketitle

\hrulefill

\textbf{Question.} Using the substitutions $u = y - x$ and $v = x + y,$ express the integral $\displaystyle\iint_D\exp\left(\dfrac{y-x}{y+x}\right)d(x, y),$ where $D$ is the triangular region with vertices $(0,0),\;(1,0)$ and $(0,1)$ in the following form:
\[\int_{a}^{b} \int_{c(v)}^{d(v)} f(u,v) du dv. \] 
% question here
\hfill [5 marks]

\hrulefill

\textbf{Answer.}\\
The given substitution can be rearranged to give $y = \frac{1}{2}(u + v)$ and $x = \frac{1}{2}(v - u).$\\
Let $\Phi:\Omega\to\mathbb{R}^2$ be defined as $\Phi(u, v) := \left(\frac{1}{2}(v - u),\;\frac{1}{2}(v + u)\right),$ where $\Omega = \mathbb{R}^2.$\\
Note that $\Phi$ is a one-to-one transformation. Moreover, if we write $\Phi = (\phi_1,\;\phi_2),$ then $\phi_1$ and $\phi_2$ have continuous partial derivatives in $\Omega.$\\
Also,
\[J(\Phi)\left(u_{0}, v_{0}\right) =\operatorname{det}\left[\begin{array}{ll}{\frac{\partial x}{\partial u}\left(u_{0}, v_{0}\right)} & {\frac{\partial x}{\partial v}\left(u_{0}, v_{0}\right)} \\ {\frac{\partial y}{\partial u}\left(u_{0}, v_{0}\right)} & {\frac{\partial y}{\partial v}\left(u_{0}, v_{0}\right)}\end{array}\right]\]
\[ = \operatorname{det}\left[\begin{array}{ll}{-1/2} & {1/2} \\ {1/2} & {1/2}\end{array}\right] = -1/2 \neq 0\]
Let $E := \{(u, v) \in \Omega : 0 \le v \le 1,\; -v \le u \le v\}.$\\
Then, $\Phi(E) = D.$ (A linear transformation with nonzero Jacobian as above sends vertices of a polygon to the vertices. Alternately, one can show that if $(u, v) \in E,$ then $\Phi(u, v) \in D$ and if $(x, y) \in D,$ then there exists $(u, v) \in E$ such that $\Phi(u, v) = (x, y).$ )\\~\\
Thus, the given integral is the same as 
\[\iint_E e^{u/v}|J(\Phi)(u, v)|d(u, v) = \iint_Ee^{u/v}\left|-\frac{1}{2}\right|d(u, v).\]
Using Fubini's theorem, we can write it in the desired form as -
\[\int_{0}^{1} \int_{-v}^{v} \frac{1}{2}e^{u/v} du dv. \]
That is, $a = 0,\;b = 1,\;c(v) = -v,$ and $d(v) = v$ where $c$ and $d$ are defined on $[0, 1].$ Moreover, $f(u, v) = \frac{1}{2}e^{u/v}$ for $(u, v) \in E.$

\hrulefill

\vspace{0.2 cm}

Broad marking scheme - 2 marks for correct Jacobian, 2 marks for correct $E,$ and 1 mark for correct final form using Fubini.

\hrulefill

\vspace{0.2 cm}

Points to be noted -
\begin{enumerate} 
	\item Note that $dx, dy, du, dv$ are not real numbers. It makes no sense to write $dy = \frac{1}{2}(du + dv)$ or any other such algebraic expression.
	\item It also makes not much sense to write $d\left(\frac{v-u}{2},\;\frac{u+v}{2}\right).$
	\item Marks have not been deducted for lack of mentioning Fubini's theorem this time but that must be kept in mind from next time onwards.
	\item Some have written the Jacobian in the opposite manner. Make note of what is to be differentiated with respect to what.
	\item Remember that the Jacobian is the determinant itself, not the matrix.
	\item When writing the integral as an integral over $E,$ the modulus of the Jacobian must be multiplied, not the Jacobian itself.
\end{enumerate}

\end{document}