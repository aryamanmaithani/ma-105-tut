\documentclass{article}
\usepackage{amsmath, amssymb, amsfonts, amsthm, mathtools}
\usepackage[utf8]{inputenc}
\usepackage[inline]{enumitem}
\usepackage{cancel}
\usepackage{soul}
\usepackage{hyperref}
\newtheorem{theorem}{Theorem}
\newtheorem{lem}{Lemma}
\newtheorem{defn}{Definition}

\setlength\parindent{0pt}

\usepackage{xcolor}
\definecolor{mybgcolor}{RGB}{46, 51, 63} %46, 51, 63

\usepackage{pagecolor}
\pagecolor{mybgcolor}
\color{white}

\usepackage{geometry}
\geometry{
	a4paper,
	total={170mm,257mm},
	left=20mm,
	top=20mm,
}

\title{Short Quiz 6: Solution}      % here
\author{Aryaman Maithani}
\date{11th September, 2019}  		 % here

\begin{document}
\maketitle

\hrulefill

\textbf{Question.} State whether the following statement is true or false. Justify your answer.
\[\int_{0}^{2\pi} e^{\sin x + \cos x} dx = \int_{\sqrt{2}-\pi}^{\sqrt{2}+\pi} e^{\sin x + \cos x} dx. \]
\hfill [5 marks]
\begin{flushright}
	[2 marks for correct alternative (T/F); 3 marks for correct justification]\\~\\
\end{flushright}

\hrulefill

\textbf{Answer.} T \hfill \textbf{[2]}\\  % Fill T/F
Justification: \\
The function $f:\mathbb{R} \to \mathbb{R}$ defined by
\[f(x) := e^{\sin x + \cos x} \quad \text{for } x \in \mathbb{R}\]
is clearly continuous and periodic with period $2\pi.$ \hfill \textbf{[1]}\\
Hence, if we let
\[F(x) := \int_{x}^{x + 2\pi} f(t)dt - \int_{0}^{x + 2\pi} f(t) dt \quad \text{for }x\in\mathbb{R}, \]
then by the Fundamental Theorem (Part 1), $F$ is differentiable and $F'(x) = f(x+2\pi) - f(x) = 0$ for all $x \in \mathbb{R}.$
\begin{flushright}
\textbf{[1]}
\end{flushright}

As $F$ is defined on an interval, we get that $F$ is a constant function, and in particular,
\[F(0) = F(\sqrt{2} - \pi),\]
which yields the desired equality. \hfill \textbf{[1]}

\hrulefill

\vspace{0.2 cm}

Points to be noted -
\begin{enumerate} 
	\item $2$ marks have been cut if the person has simply used the property that we had proven for integrals of continuous and periodic functions.
	\item Further half a mark has been deducted if the person has simply written that the integrand is periodic without mentioning its continuity.
	\item A mark has been deducted has been deducted for those who differentiated $F$ without justifying \emph{why} it's differentiable. (The integrand is continuous.)
	\item A person may get the correct answer without invoking continuity but using a different proof than the one given which relies only on periodicity. Full marks have been awarded if that is done correctly.
	\item No marks have been deducted even if the student has not mentioned that $F$ is defined on an interval.
\end{enumerate}

\end{document}