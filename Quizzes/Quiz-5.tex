\documentclass{article}
\usepackage{amsmath, amssymb, amsfonts, amsthm, mathtools}
\usepackage[utf8]{inputenc}
\usepackage[inline]{enumitem}
\usepackage{cancel}
\usepackage{soul}
\usepackage{hyperref}
\newtheorem{theorem}{Theorem}
\newtheorem{lem}{Lemma}
\newtheorem{defn}{Definition}

\setlength\parindent{0pt}

\usepackage{xcolor}
\definecolor{mybgcolor}{RGB}{46, 51, 63} %46, 51, 63

\usepackage{pagecolor}
\pagecolor{mybgcolor}
\color{white}

\usepackage{geometry}
\geometry{
	a4paper,
	total={170mm,257mm},
	left=20mm,
	top=20mm,
}

\title{Short Quiz 5: Solution}      % here
\author{Aryaman Maithani}
\date{4th September, 2019}  		 % here

\begin{document}
\maketitle

\hrulefill

\textbf{Question.} State whether the following statement is true or false. Justify your answer.\\ 
If the $n$th Taylor polynomial of $f:\left(-\pi/2, \pi/2\right) \to \mathbb{R}$ defined by $f(x) = \tan x$ around $0$ is given by
\[a_0 + a_1x + a_2x^2 + \cdots a_nx^n\]
then $3a_3 + 1 = 0.$
\hfill [5 marks]
\begin{flushright}
	[2 marks for correct alternative (T/F); 3 marks for correct justification]\\~\\
\end{flushright}

\hrulefill

\textbf{Answer.} F \hfill \textbf{[2]}\\  % Fill T/F
Justification: Clearly, derivatives of $f$ of all orders exist on $(-\pi/2, \pi/2)$ and so
\[a_k = \frac{f^{(k)}(0)}{k!}\]
\begin{flushright}
	[1]
\end{flushright}
Thus, to determine $a_3,$ we compute $f'''(0).$ We have
\begin{align*}
	f'(x) &= \sec^2x,\\
	f''(x) &= 2\sec x(\sec x\tan x) = 2\sec^2 x\tan x, \text{ and }\\
	f'''(x) &= 4\sec x(\sec x\tan x)\tan x + 2\sec^2 x(\sec^2 x) = 4\sec^2x\tan^2x + 2\sec^4x
\end{align*}
\begin{flushright}
	[1]
\end{flushright}
Consequently,
\[f'''(0) = 2 \text{ and } a_3 = \frac{f'''(0)}{3!} = \frac{2}{6} = \frac{1}{3}.\]
\begin{flushright}
	[1]
\end{flushright}
Thus, $3a_3 + 1 = 2 \neq 0.$ 

\hrulefill

\vspace{0.2 cm}

Points to be noted -
\begin{enumerate} 
	\item If you have arrived at $a_3 = 1/3$ by equating $f(x)$ with its Taylor polynomial and then differentiating twice and putting $x = 0,$ then 2 marks have been deducted.\\
	This is incorrect because a function is not necessarily equal to its $n$th Taylor polynomial. There is also the remainder term.\\
	Moreover, the remainder itself is not a constant in general.
	\item For those who have done the above but by taking an infinite sum, similar considerations apply as you have not argued about convergence of any sort.\\
	Moreover, you have not shown that an infinite series can be differentiated term by term.
\end{enumerate}

\end{document}