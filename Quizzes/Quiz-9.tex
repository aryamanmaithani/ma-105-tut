\documentclass{article}
\usepackage{amsmath, amssymb, amsfonts, amsthm, mathtools}
\usepackage[utf8]{inputenc}
\usepackage[inline]{enumitem}
\usepackage{cancel}
\usepackage{soul}
\usepackage{hyperref}
\newtheorem{theorem}{Theorem}
\newtheorem{lem}{Lemma}
\newtheorem{defn}{Definition}

\setlength\parindent{0pt}

\usepackage{xcolor}
\definecolor{mybgcolor}{RGB}{46, 51, 63} %46, 51, 63

\usepackage{pagecolor}
\pagecolor{mybgcolor}
\color{white}

\usepackage{geometry}
\geometry{
	a4paper,
	total={170mm,257mm},
	left=20mm,
	top=20mm,
}

\title{Short Quiz 9: Solution}      % here
\author{Aryaman Maithani}
\date{9th October, 2019}  		 % here

\begin{document}
\maketitle

\hrulefill

\textbf{Question.} \\
Let $f, g : \mathbb{R}^2 \to \mathbb{R}$ be defined by
\[f(x, y) := \sqrt{x^2 + y^2} \text{ and } g(x, y) := |xy| \text{ for } (x, y) \in \mathbb{R}^2.\]
Determine which of the following statements is true. Justify your answer.
\begin{enumerate}[label = (\alph*)] 
 	\item Both $f$ and $g$ are differentiable at $(0, 0).$
 	\item $f$ is differentiable at $(0, 0),$ but $g$ is not differentiable at $(0, 0).$
 	\item $g$ is differentiable at $(0, 0),$ but $f$ is not differentiable at $(0, 0).$
 	\item Neither $f$ nor $g$ is differentiable at $(0, 0).$
 \end{enumerate} 
 \hfill [5 marks]
\begin{flushright}
	[2 marks for correct alternative (T/F); 3 marks for correct justification]\\~\\
\end{flushright}

\hrulefill

\textbf{Answer.} The correct alternative is (c). \hfill \textbf{[2]}\\ 
Justification: \\
Note that for $h \neq 0,$ one has
\[\dfrac{f(0 + h, 0) - f(0, 0)}{h} = \dfrac{|h|}{h}.\]
Hence, $f_x(0, 0)$ does not exist as $\displaystyle\lim_{h\to 0}\dfrac{|h|}{h}$ does not. \hfill \textbf{[1]}\\~\\
%
For $(h, k) \neq (0, 0)$ note that $|h| \le \sqrt{h^2 + k^2}$ and hence,
\[\dfrac{g(0 + h, 0 + k) - g(0, 0) - 0\cdot h - 0\cdot k }{\sqrt{h^2 + k^2}} = \dfrac{|hk|}{\sqrt{h^2 + k^2}} \le |k|.\]
Hence, we get that \[\lim_{(h, k)\to (0, 0)}\dfrac{g(0 + h, 0 + k) - g(0, 0) - 0\cdot h - 0\cdot k }{\sqrt{h^2 + k^2}} = 0.\]
Thus, $g$ is differentiable at $(0, 0)$ with (total) derivative $(0, 0).$ \hfill \textbf{[2]}

\hrulefill

\vspace{0.2 cm}

Points to be noted -
\begin{enumerate} 
	\item Simply calculating $f_x(x_0, y_0)$ for $(x_0, y_0) \neq (0, 0)$ and then concluding that $f_x(0, 0)$ does not exist by showing that $\displaystyle\lim_{(x_0, y_0)\to (0, 0)}f_x(x_0, y_0)$ does not exist is not correct as limit of derivative not existing does not imply non-existence of derivative of the limit. Hence, no marks for this justification has been given.\\
	This had been mentioned in the last quiz as well.
	\item Many have incorrectly tried to invoke the sufficient condition for differentiability. Note that $g_x(0, k)$ does not even exist for $k \neq 0$ and hence, $g_x$ isn't continuous at $(0, 0).$ Similar argument for $g_y$ as well.
	\item Some have concluded the differentiability of $g$ by incorrect arguments such as the following:
	\begin{enumerate}[nosep] 
		\item Existence of all directional derivatives.
		\item Existence of partial derivatives and continuity of $g.$
		\item $(\mathbf{D_u}g)(0, 0) = (\nabla g)(0, 0)\cdot \mathbf{u}$ being true for all unit vectors $\mathbf{u}\in\mathbb{R}^2.$
	\end{enumerate}
	Note that the above are simply necessary conditions but not sufficient. Indeed, consider the function $h:\mathbb{R}^2 \to \mathbb{R}$ defined as
	\[h(x, y) := \left\{
	\begin{array}{c c}
		\dfrac{x^3y}{x^4 + y^2} & (x, y) \neq (0, 0)\\
		0 & (x, y) = (0, 0)	
	\end{array}
	\right.\]
	One can verify that the above function satisfies all the condition listed earlier but is not differentiable at $(0, 0).$
\end{enumerate}
\end{document}